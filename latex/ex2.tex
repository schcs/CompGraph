\documentclass{amsart}


\usepackage{verbatim}
\usepackage{amssymb}

\usepackage[brazilian]{babel}
%\usepackage[latin1]{inputenc}
%\usepackage[T1]{fontenc}

\newcommand{\sym}[1]{{\mathsf S}_{#1}}
\newcommand{\alt}[1]{{\mathsf A}_{#1}}
\newcommand{\SL}[2]{{\sf SL}(#1,#2)}
\newcommand{\GL}[2]{{\sf GL}(#1,#2)}
\newcommand{\N}{\mathbb N}
\newcommand{\F}{\mathbb F}
\newcommand{\Z}{\mathbb Z}
\newcommand{\Q}{\mathbb Q}
\newcommand{\R}{\mathbb R}
\newcommand{\C}{\mathbb C}
\newcommand{\nullvec}{\overline 0}

\newcommand{\mat}[2]{{\mathsf M}_{#1,#2}}

\oddsidemargin 0pt
\evensidemargin 0pt
\textheight 8.1in \textwidth 6.3in


\relpenalty=10000
\binoppenalty=10000
\tolerance=500

\parskip 5pt
\parindent 0pt

\begin{document}


\begin{center}
\large Geometria com aplicações na gráfica computacional\\
{\bf\large Folha 2 de exercícios}\\
csaba@mat.ufmg.br
\end{center}


\bigskip

{\bf 1.} Seja $V$ um espaço vetorial de dimensão finita, seja $\mathcal T(V)$ o grupo de translações de $V$, seja $\mbox{GL}(V)$ o grupo de transformações lineares invertíveis de $V$ e seja $\mbox{End}(V)$ o conjunto de transformações lineares (não necessariamente invertíveis) de $V$.  
\begin{enumerate}
    \item Mostre que $\mathcal T(V)\mbox{GL}(V)=\mbox{GL}(V)\mathcal T(V)$.
    \item Mostre que $\mathcal T(V)\mbox{End}(V)\neq \mbox{End}(V)\mathcal T(V)$.
\end{enumerate}
[Dica: Na parte~(2), considere a translação $T$ por $(1,1)$ e a projeção $P:(x,y)\mapsto (0,y)$
sobre $\R^2$. Mostre que 
$P\circ T$ não pode ser escrito na forma $T'\circ X$ com $T'\in \mathcal T(V)$ e $X\in\mbox{End}(V)$.]

\medskip

{\bf 2.} Seja $X$ um elemento de $\mbox{AGL}(\R^2)=\mathcal T(\R^2)\mbox{GL}(\R^2)$ tal que 
\[
    X(0)=(2,3),\quad X(1,0)=(-3,3),\quad X(0,1)=(-4,2).
\]
Escreva $X$ na forma $T_{t_1}Y$ e também na forma $ZT_{t_2}$ onde $Y,Z\in\mbox{GL}(\R^2)$ e 
$T_{t_1},T_{t_2}\in\mathcal T(\R^2)$. Qual é a relação entre $T_{t_1}$ e $T_{t_2}$ e entre $Y$ e $Z$.

    \medskip

{\bf 3.} Seja $R_\alpha:\R^2\to\R^2$ uma rotação em $\R^2$ pelo ângulo 
$\alpha$ ao redor da origem. Calcule os autovalores complexos $R_\alpha$
e os autovetores correspondentes.

\medskip


{\bf 4.} Seja $T:\R^n\to\R^n$ uma transformação ortogonal e seja $\alpha\in\C$ um autovalor de $T$. 
\begin{enumerate}
    \item Mostre que $\det T\in\{1,-1\}$.
    \item Mostre que $|\alpha|=1$.
    \item Deduza que se $\alpha\in\R$, então $\alpha\in\{1,-1\}$. 
    \item Mostre que o conjugado complexo $\bar \alpha$ é também um autovalor de $T$.
    \item Deduza que se $n$ for impar, então $\det T$ é autovalor de $T$. 
\end{enumerate}

\medskip

{\bf 5.} Seja $T_{(1,1)}:\R^2\to \R^2$ a translação pelo vetor $(1,1)$ e $R_\alpha:\R^2\to\R^2$ a rotação por um ângulo $\alpha$ ao redor da origem. Como pode caraterizar a transformação $TR_\alpha T^{-1}$?

\medskip


{\bf 6.} Seja $X$ a reflexão de $\R^2$ em torno da reta com equação $x-y+1=0$. Escreva $X$ na forma 
$T_{t_1}Y$ e também na forma $ZT_{t_2}$ onde $Y,Z\in\mbox{GL}(V)$ e $T_{t_1},T_{t_2}\in\mathcal T(\R^2)$. 
\end{document}

