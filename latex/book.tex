\documentclass[12pt]{amsbook}
\usepackage{amsthm}
\usepackage{amssymb}
\usepackage{showkeys}
%\usepackage{tikz-cd}

\renewcommand{\a}{\mathfrak a}
\renewcommand{\b}{\mathfrak b}
\newcommand{\m}{\mathfrak m}
\newcommand{\n}{\mathfrak n}
\newcommand{\p}{\mathfrak p}
\newcommand{\q}{\mathfrak q}
\renewcommand{\r}{\mathfrak r}
\newcommand{\F}{\mathbb F}
\renewcommand{\L}{\mathbb L}
\newcommand{\Q}{\mathbb Q}
\newcommand{\N}{\mathbb N}
\newcommand{\Z}{\mathbb Z}
\newcommand{\C}{\mathbb C}
\newcommand{\Ha}{\mathbb H}


\newcommand{\K}{\mathbb K}
\newcommand{\R}{\mathbb R}
\newcommand{\fracf}[1]{\mbox{Frac}(#1)}
\newcommand{\spec}[1]{\mbox{Spec}(#1)}
\newcommand{\len}{\mbox{len}\,}
\newcommand{\id}{\mbox{id}\,}
\DeclareMathOperator{\sen}{\textrm{\rm sen}}
%\newcommand{\sen}{\mbox{sen}\,} 
\newcommand{\gl}[1]{\mbox{GL}_{#1}}
\newcommand{\SL}[1]{\mbox{SL}_{#1}}
\newcommand{\glv}[1]{\mbox{GL}(#1)}
\newcommand{\slv}[1]{\mbox{SL}(#1)}
\newcommand{\Sym}{\mbox{Sym}}
\usepackage[portuguese]{babel}
\newcommand{\rot}[1]{{\textrm{Rot}}_{#1}}
\newcommand{\refl}[1]{{\textrm{Ref}}_{#1}}




\newtheorem{theorem}{Teorema}[chapter]
\newtheorem{corollary}[theorem]{Corolário}
\newtheorem{lemma}[theorem]{Lema}
\newtheorem{exercise}[theorem]{Exercício}

\theoremstyle{definition}
\newtheorem{example}[theorem]{Exemplo}
\newtheorem{definition}[theorem]{Definição}
\renewcommand{\P}{\mathcal P}
\renewcommand{\L}{\mathcal L}
\newcommand{\I}{\mathcal I}
\newcommand{\PP}{\mathbb P}
\newcommand{\agl}{\mbox{AGL}}





\oddsidemargin 0pt
\evensidemargin 0pt
\textheight 8.1in \textwidth 6.3in


\relpenalty=10000
\binoppenalty=10000
\tolerance=500


\begin{document}
\title{Álgebra e Geometria com Aplicações na Computação Gráfica}
\author{Csaba Schneider}
\maketitle
\tableofcontents

\documentclass[12pt]{amsart}
\usepackage{amsthm}
\usepackage{amssymb}
\usepackage{showkeys}
\usepackage{tikz-cd}
\usepackage[portuguese]{babel}

\renewcommand{\a}{\mathfrak a}
\renewcommand{\b}{\mathfrak b}
\newcommand{\m}{\mathfrak m}
\newcommand{\n}{\mathfrak n}
\newcommand{\p}{\mathfrak p}
\newcommand{\q}{\mathfrak q}
\renewcommand{\r}{\mathfrak r}
\newcommand{\F}{\mathbb F}
\renewcommand{\L}{\mathbb L}
\newcommand{\Q}{\mathbb Q}
\newcommand{\N}{\mathbb N}
\newcommand{\Z}{\mathbb Z}


\newcommand{\K}{\mathbb K}
\newcommand{\R}{\mathbb R}
\newcommand{\fracf}[1]{\mbox{Frac}(#1)}
\newcommand{\spec}[1]{\mbox{Spec}(#1)}
\newcommand{\len}{\mbox{len}\,}
\newcommand{\id}{\mbox{id}\,}
\newcommand{\sen}{\mbox{sen}\,}

\newtheorem{theorem}{Teorema}
\newtheorem{corollary}{Corolário}[theorem]
\newtheorem{lemma}[theorem]{Lema}
\newtheorem{exercise}[theorem]{Exercício}

\theoremstyle{definition}
\newtheorem{example}[theorem]{Exemplo}
\newtheorem{definition}[theorem]{Definição}


\oddsidemargin 0pt
\evensidemargin 0pt
\textheight 8.1in \textwidth 6.3in


\relpenalty=10000
\binoppenalty=10000
\tolerance=500


\begin{document}

\title{Transformações lineares}
\maketitle

\section{Coordenadas}
Seja $V$ um espaço vetorial de dimensão finita. Seja $B=\{b_1,\ldots,b_n\}$ uma base de $V$ e seja 
$v\in V$. Então $v$ pode ser escrito unicamente como 
\[
    v=\alpha_1b_1+\cdots+\alpha_nb_n.
\]
O vetor $[v]_B=(\alpha_1,\ldots,\alpha_n)$ chama se \emph{o vetor das coordenadas} de $v$ na base $B$.

\begin{example}
    Seja $V=\{(x,y,z)\in\R^3\mid x+y+z=0\}$ e seja $B=\{b_1=(1,0,-1),b_2=(1,-1,0)\}$ (verifique que $V$ é 
    espaço vetorial com base $B$). Ponha $v=(3,2,-5)$. Então 
    \[
        v=5b_1-2b_2,
    \]
    e assim $[v]_B=(5,-2)$. 
\end{example}

\begin{exercise}
    Verifique que a aplicação $V\to \R^n$ definida por $v\mapsto [v]_B$ é um isomorfismo linear (ou seja, uma aplicação linear 
    injetiva e sobrejetiva) 
\end{exercise}

\section{A matriz de uma transformação linear}
Seja $T:V\to W$ uma transformação linear entre dois espaços vetoriais de dimensão finita. Assuma que 
$B=\{b_1,\ldots,b_n\}$ é uma base de $V$, enquanto $C=\{c_1,\ldots,c_m\}$ é uma base de $W$. 
Como $T(b_i)\in W$,  o vetor $T(b_i)$ pode ser escrito como 
\[
    T(b_i)=\alpha_{i,1}c_1+\ldots+\alpha_{i,m}c_m
\]
com $\alpha_{i,j}\in \R$. Nós definimos a matriz de $T$ relativa às bases $B$ e $C$ como 
\[
    [T]_C^B=\begin{pmatrix} \alpha_{1,1} & \cdots & \alpha_{n,1}\\
        \vdots & \ddots & \vdots \\
        \alpha_{1,m} & \cdots & \alpha_{n,m}
    \end{pmatrix}=\left([T(b_1)]_C,\ldots,[T(b_n)]_C\right).
\]
Ou seja, a matriz $[T]^B_C$ contém os vetores $[T(b_i)]_C$ nas suas colunas. A matriz $[T]^B_C$ é uma matriz
$m\times n$. 

\begin{example}
Considere a transformação $T:\R^3\to V$, $T(x,y,z)=(x-y,y-z,z-x)$ onde $V$ é o mesmo espaço que no exemplo anterior. Seja $B$ a base canônica de $\R^3$ e $C=\{c_1=(1,0,-1),c_2=(0,1,-1)\}$. Então temos que 
\begin{align*}
    T(1,0,0)&=(1,0,-1)=c_1\\
    T(0,1,0)&=(-1,1,0)=-c_1+c_2\\
    T(0,0,1)&=(0,-1,1)=-c_2.
\end{align*}
Logo
\[
    [T]^B_C=\begin{pmatrix} 1 & -1 & 0 \\ 0 & 1 & -1\end{pmatrix}
\]
\end{example}
\begin{lemma}\label{lem:matr}
    Usando a notação no parágrafo anterior, temos que 
    \[
        [T(v)]_C=[T]^B_C\cdot [v]_B.
    \]
\end{lemma}

Note que no lado direito da equação no Lema~\ref{lem:matr}, o vetor $[v]_B$ é visto como vetor coluna 
para a multiplicação fazer sentido. Isso poderia ser denotado por $[v]_B^t$, mas nós escolhemos a 
notação mais simples. 

\begin{proof}
    Primeiro assuma que $v=b_i\in B$. Então $[T(b_i)]_C$ é justamente a $i$-ésima coluna de $[T]^B_C$ 
    e $[b_i]_B$ é o $i$-ésimo vetor na base canônica de $\R^m$. 
    Logo temos obviamente que $[T(b_i)]_C=[T]^B_C\cdot [b_i]_B$. Quando $v\in V$ é arbitrário, escreva que 
    \[
        v=\beta_1b_1+\cdots+\beta_n b_n;
    \]
    ou seja, $[v]_B=(\beta_1,\ldots,\beta_n)$. Ora,
    \begin{align*}
        [T(v)]_C&=[T(\beta_1b_1+\cdots+\beta_n b_n)]_C\\&=
        \beta_1 [T(b_1)]_C+\cdots +\beta_n [T(b_n)]_C\\&=
        \beta_1 [T]^B_C\cdot e_1+\cdots +\beta_n[T]^B_C\cdot e_n\\&=
        [T]^B_C \cdot [v]_B
    \end{align*}
    onde $e_1,\ldots,e_n$ são os vetores (colunas) da base canônica de $\R^n$.
\end{proof}

\section{Mudança de base}
Seja $V$ um espaço vetorial com duas bases $B=\{b_1,\ldots,b_n\}$ e $C=\{c_1,\ldots,c_n\}$. A transformação 
$\id:V\to V$, $\id(v)=v$  é linear e podemos considerar a sua matriz $[\id]_B^C$. Pelo que fizemos 
nas seções anteriores
\[
    [\id]_B^C=\begin{pmatrix} \alpha_{1,1} & \cdots & \alpha_{n,1}\\
        \vdots & \ddots & \vdots \\
        \alpha_{1,m} & \cdots & \alpha_{n,m}
    \end{pmatrix}=\left([c_1]_B,\ldots,[c_n]_B\right)
\]
onde os coeficientes estão determinados pelas equações 
\[
    c_i=\alpha_{i,1}b_1+\cdots+\alpha_{i,n}b_n.
\]
A matriz $[\id]_B^C$ chama-se \emph{matriz mudança de base} (de $B$ para $C$).

\begin{lemma}
    Usando a notação no parágrafo anterior, temos que 
    \[
        [v]_B=[\id]_B^C\cdot [v]_C.
    \]
\end{lemma}
\begin{proof}
    Segue do Lema~\ref{lem:matr}.
\end{proof}

\begin{exercise}
    Demonstre que $[\id]_C^B=([id]_B^C)^{-1}$. 
\end{exercise}

\begin{example}
    Seja $V=\R^2$, $B=\{e_1,e_2\}$ (a base canônica), e $C=\{c_1=(1,1),c_2=(1,-1)\}$. Logo
    \[
        [\id]_B^C=\begin{pmatrix} 1 & 1 \\ 1 & -1\end{pmatrix}\quad \mbox{e}\quad 
        [\id]_C^B=\left([\id]_B^C\right)^{-1}=\frac 12[\id]_B^C.
    \] 
    Seja $v=(-1,2)$. Então $[v]_B=(-1,2)$ e 
    \[
        [v]_C=[\id]^B_C[v]_B=(1/2,-3/2).
    \]
    De fato $v=(1/2)c_1-(3/2)c_2$.
\end{example}

\begin{exercise}\label{ex:comp}
    Sejam $T_1:V\to U$ e $T_2:U\to W$ transformações lineares, e sejam $B$, $C$, e $D$ bases de 
    $V$, $U$, e $W$, respetivamente. Mostre que 
    \[
    [T_2\circ T_1]^B_D=[T_2]^C_D\cdot [T_1]^B_C.
\]
\end{exercise}


\section{Transformações lineares e mudança de base}

Seja $T:V\to W$ uma transformação linear entre os espaços $V$ e $W$ de dimensão finita. Sejam 
$B$, $B'$ bases de $V$ e $C$, $C'$ bases de $W$. 

\begin{lemma}
    Temos que 
    \[
        [T]_{C'}^{B'}=[\id_W]_{C'}^C\cdot [T]^B_C\cdot [\id_V]_B^{B'}.        
    \]
\end{lemma}
\begin{proof}
    Aplique o Exercício~\ref{ex:comp}.
\end{proof}

Quando $T:V\to V$ é um endomorfismo, nós geralmente calculamos a matriz $[T]_B^B$. Se $B$ e $C$ são duas bases de $T$, então temos que 
\[
    [T]_C^C=[\id]_C^B\cdot [T]_B^B\cdot [\id]_B^C=[\id]_C^B\cdot [T]_B^B\cdot([\id]_C^B)^{-1}.
\]

Note que se $Y$ é uma matriz e $X$ é uma matriz invertível $n\times n$, então diz-se que a 
 matriz $XYX^{-1}$ é um conjugada de $Y$. 

 \begin{exercise}
    Sejam $Y_1$ e $Y_2$ matrizes conjugadas. Demonstre as seguintes afirmações. 
    \begin{enumerate}
        \item $\det Y_1=\det Y_2$.
        \item $Y_1$ e $Y_2$ têm os mesmos autovalores.
        \item Seja $Y_2=XY_1X^{-1}$ e seja $v\in\R^n$. Então $v$ é um autovetor de $Y_1$ se e somente se 
        $Xv$ é autovetor de $Y_2$. Além disso $v$ e $Xv$ correspondem ao mesmo autovalor.
    \end{enumerate}
 \end{exercise}

\section{Um exemplo detalhado: As reflexões}

Assuma que $t=(a,b)\in\R^2$ é um vetor com $||t||=\sqrt{a^2+b^2}=1$. Define 
\[
    R_t:\R^2\to \R^2,\quad R_t(v)=v-2(v\cdot t)t
\] 
onde $v\cdot t$ denota o produto escalar entre $v$ e $t$. É fácil verificar que $R_t$ é linear. 
Seja $t'=(b,-a)$ um vetor normal (ortoginal) ao vetor $t$. Então temos que $t\cdot t=1$ e $t\cdot t'=0$  
e assim 
\[
    R_t(t)=-t\quad\mbox{enquanto}\quad R_t(t')=t'.
\] 
Como vetores $t$  e $t'$ formam uma base $C$, faz sentido perguntar a matriz de $R_t$ nesta base. De fato 
\[
    [R_t]_C^C=\begin{pmatrix} -1 & 0 \\ 0 & 1\end{pmatrix}.
\]  
Seja $B$ a base canônica de $\R^2$. Então temos que 
\[
    [\id]_B^C=\begin{pmatrix} a & b \\ b & -a \end{pmatrix}.
\]
Além disso, $[\id]_B^C$ é uma matriz ortogonal simêtrica, e assim $[\id]_C^B=([\id]_B^C)^{-1}=[\id]_B^C$. 
Logo 
\[
    [R_t]_B^B=[\id]_B^C \cdot [R_t]_C^C\cdot [\id]_C^B=
    \begin{pmatrix} a & b \\ b & -a \end{pmatrix}\begin{pmatrix} -1 & 0 \\ 0 & 1\end{pmatrix}
    \begin{pmatrix} a & b \\ b & -a \end{pmatrix}=\begin{pmatrix}-a^2+b^2 & -2ab \\ -2ab & a^2-b^2
    \end{pmatrix}.
\]
Alternativamente, podemos verificar com conta direta que 
\[
    R_t(1,0)=(1-2a^2,-2ab)\quad\mbox{e}\quad R_t(0,1)=(-2ab,1-2b^2)
\]
e que 
\[
    [R_t]_B^B=\begin{pmatrix} 1-2a^2 & -2ab \\ -2ab & 1-2b^2\end{pmatrix}.
\]
Como $a^2+b^2=1$ as duas matrizes que obtivemos para $[R_t]_B^B$ são de fato iguais. 

Usando que $a^2+b^2=1$, podemos escrever $a=\cos\alpha$ e $b=\sen\alpha$ com algum ângulo $\alpha\in[0,2\pi]$.
Assim obtemos que 
\[ 
    [R_t]_B^B=\begin{pmatrix} -\cos^2\alpha +\sen^2\alpha & -2\cos \alpha\cdot \sen\alpha\\ 
        -2\cos \alpha\cdot \sen\alpha & \cos^2\alpha-\sen^2\alpha\end{pmatrix}
        =\begin{pmatrix}-\cos(2\alpha) & -\sen(2\alpha) \\ -\sen(2\alpha) & \cos(2\alpha)
        \end{pmatrix}.
\]
Seja $\alpha=\alpha'+\pi/2$. Com $\alpha'$ podemos escrever $[R_t]_B^B$ na forma ainda mais simples 
como 
\[
    [R_t]_B^B=\begin{pmatrix}\cos(2\alpha') & \sen(2\alpha') \\ \sen(2\alpha') & -\cos(2\alpha').
    \end{pmatrix}
\]
\end{document}

\documentclass[12pt]{amsart}
\usepackage{amsthm}
\usepackage{amssymb}
\usepackage{showkeys}
\usepackage{tikz-cd}

\renewcommand{\a}{\mathfrak a}
\renewcommand{\b}{\mathfrak b}
\newcommand{\m}{\mathfrak m}
\newcommand{\n}{\mathfrak n}
\newcommand{\p}{\mathfrak p}
\newcommand{\q}{\mathfrak q}
\renewcommand{\r}{\mathfrak r}
\newcommand{\F}{\mathbb F}
\renewcommand{\L}{\mathbb L}
\newcommand{\Q}{\mathbb Q}
\newcommand{\N}{\mathbb N}
\newcommand{\Z}{\mathbb Z}
\newcommand{\C}{\mathbb C}


\newcommand{\K}{\mathbb K}
\newcommand{\R}{\mathbb R}
\newcommand{\fracf}[1]{\mbox{Frac}(#1)}
\newcommand{\spec}[1]{\mbox{Spec}(#1)}
\newcommand{\len}{\mbox{len}\,}
\newcommand{\id}{\mbox{id}\,}
\newcommand{\sen}{\mbox{sen}\,}
\newcommand{\gl}[1]{\mbox{GL}_{#1}}
\newcommand{\SL}[1]{\mbox{SL}_{#1}}
\newcommand{\glv}[1]{\mbox{GL}(#1)}
\newcommand{\slv}[1]{\mbox{SL}(#1)}
\newcommand{\Sym}{\mbox{Sym}}
\usepackage[portuguese]{babel}

\newtheorem{theorem}{Teorema}
\newtheorem{corollary}{Corolário}[theorem]
\newtheorem{lemma}[theorem]{Lema}
\newtheorem{exercise}[theorem]{Exercício}

\theoremstyle{definition}
\newtheorem{example}[theorem]{Exemplo}
\newtheorem{definition}[theorem]{Definição}




\oddsidemargin 0pt
\evensidemargin 0pt
\textheight 8.1in \textwidth 6.3in


\relpenalty=10000
\binoppenalty=10000
\tolerance=500


\begin{document}

\title{Os movimentos do plano}
\maketitle

\section{Transformações}

Seja $\Omega$ um conjunto. Uma aplicação $f:\Omega\to \Omega$ chama-se uma \emph{transformação} de $\Omega$. A transformação $f$ é dita \emph{injetiva} se $f(v)=f(w)$ implica $v=w$ para todo $v,w\in\Omega$;
$f$ chama-se \emph{sobrejetiva} se para todo $w\in\Omega$ existe $v\in\Omega$ tal que $f(v)=w$. A transformação
$f$ chama-se \emph{bijetiva} ou \emph{invertível} se ela é injetiva e sobrejetiva. Se $f$ é uma transformação 
invertível, então existe a sua inversa $f^{-1}:\Omega\to\Omega$ definida pela regra que 
$f(v)=w$ se e somente se $f^{-1}(w)=v$ para todo $v,w\in\Omega$.  

As transformações de $\Omega$ podem ser compostas. Se $f,g:\Omega\to \Omega$ então $f\circ g:\Omega\to \Omega$ 
é definida como $(f\circ g)(v)=f(g(v))$ para todo $v\in \Omega$. A composição de transformações é associativa no sentido que $(f\circ g)\circ h=f\circ(g\circ h)$ para todo $f,g,h:\Omega\to \Omega$. 

Nós vamos estudar principalmente as transformações do plano $\R^2$ e o espaço $\R^3$. 

\begin{example}
    Todo conjunto $\Omega$ tem a transformação identidade $\id_\Omega:\Omega\to \Omega$, $v\mapsto v$ 
    para todo $v\in\Omega$. Se $f:\Omega\to\Omega$ é uma transformação invertível, então 
    $f\circ f^{-1}=f^{-1}\circ f=\id_\Omega$. 
\end{example}

\begin{example}
    Seja $\Omega=\R^n$ com $n\geq 1$. Uma transformação $T:\R^n\to\R^n$ é dita linear se 
    \[
        T(\alpha v+\beta w)=\alpha T(v)+\beta T(w)
    \]
    para todo $v,w\in\R^n$ e $\alpha,\beta\in\R$. Transformações lineares são estudadas em álgebra linear.
    Uma transformação linear $T:\R^n\to \R^n$ é invertível se e somente se 
    \[
        \ker T=\{v\in \R^n\mid T(v)=0\}=\{0\}.
    \]
\end{example}

\begin{example}
    Seja $t\in\R^n$ e considere a transformação $T_t:\R^n\to \R^n$ definido como 
    \[
        T_t(v)=v+t.
    \]
    A transformação $T_t$ é chamado \emph{a translação de $\R^n$} pelo vetor $t$. Note que se 
    $t\neq 0$, então $T_t$ não é linear, pois $T_t(0)=t\neq 0$. A transformação $T_t$ é invertível e 
    $T_t^{-1}=T_{-t}$. 
\end{example}

\section{Grupos}
Seja $G$ um conjunto não vazio com uma operação que pode ser denotada por $\cdot$ (ou por $+$, ou simplesmente 
por concatenação). Isso quer dizer que com cada par de elementos 
$a,b\in G$ associamos um elemento $a\cdot b\in G$. O conjunto $G$ considerado com a operação $\cdot$ é dito 
\emph{grupo} se as seguintes propriedades estão válidas para todo $a,b,c\in G$. 
\begin{enumerate}
    \item A operação $\cdot$ é associativa; ou seja $a\cdot(b\cdot c)=(a\cdot b)\cdot c$. 
    \item Existe identidade $1\in G$ tal que $1\cdot a=a\cdot 1=a$.
    \item Todo elemento $a\in G$ possui inverso $a^{-1}$ que satisfaz $a\cdot a^{-1}=a^{-1}a=1$. 
\end{enumerate} 
Um grupo $G$ é dito \emph{abeliano} ou \emph{comutativo} se $ab=ba$ para todo $a,b\in G$. 


\begin{example}
    O conjunto $\Z$ dos inteiros é um grupo abeliano com a operação de adição. 
    A mesma coisa vale para $\Q$, $\R$, $\C$. Se $V$ é um espaço vetorial, então $V$ é um grupo abeliano
    com a operação de adição. 
\end{example}

Nós vamos considerar dois tipos de grupos, nomeadamente grupos de transformações e grupos de matrizes.

\begin{example}
    Seja $n\geq 1$, e seja $G$ um conjunto não vazio de matrizes invertíveis $n\times n$ tal que $G$ é fechado para multiplicação e se $X\in
    G$, então     $X^{-1}\in G$. Então $G$ é um grupo com a multiplicação matricial. Um tal grupo $G$ é chamado um \emph{grupo de matrizes} ou \emph{grupo matricial}. 
    Os primeiros exemplos de grupos matriciails são
    \begin{align*}
        \gl n&=\{X\mbox{ é matriz $n\times n$}\mid X\mbox{ é invertível}\}\\
        \SL n&=\{X\in\gl n\mid \det X=1\}.
    \end{align*}
    Os conjuntos $\gl n$ e $\SL n$ são grupos. É óbvio que $\SL n\subseteq \gl n$ e neste caso dizemos que 
    $\SL n$ é um subgrupo de $\gl n$ e escrevemos que $\SL n\leq \gl n$. 
\end{example}

\begin{example}
    Seja $G$ um conjunto de transformações invertíveis de um conjunto $\Omega$ tal que $G$ é fechado para 
    a composição e $T^{-1}\in G$ sempre quando $T\in G$. Neste caso $G$ é um grupo. Tal grupo chama-se 
    um \emph{grupo de transformações}. Por exemplo seja $\Omega=V$ um espaço vetorial de dimensão finita
    e considere 
    \begin{align*}
        \Sym(V)&=\{f:V\to V\mid f \mbox{ é injetiva}\}\\
        \glv V&=\{T:V\to V\mid \mbox{$T$ é linear e invertível}\}\\
        \slv V&=\{T\in \glv V\mid \det T=1\}.
    \end{align*}
    O conjunto $\mathcal T(V)=\{T_t\mid t\in V\}$ de um espaço vetorial $V$ é um grupo pois 
    $T_{t_1}\circ T_{t_2}=T_{t_1+t_2}$ e $T_{t}^{-1}=T_{-t}$ (ou seja este conjunto é fechado para  
    a composição e para os inversos). Como 
    \[
        T_{t_1}\circ T_{t_2}=T_{t_1+t_2}=T_{t_2+t_1}=T_{t_2}\circ T_{t_1},
    \]
    temos que $\mathcal T(V)$ é um grupo abeliano.
\end{example}

\begin{lemma}
    Seja $t\in V$ e $X\in \glv V$. Então $XT_tX^{-1}=T_{X(t)}$.  
\end{lemma}
\begin{proof}
    Seja $v\in V$ e computemos que 
    \[
        XT_tX^{-1}(v)=XT_t(X^{-1}(v))=X(X^{-1}(v)+t)=v+X(t)=T_{X(t)}(v).
    \]
\end{proof}

\begin{theorem}
    Assuma que $G$ é um subgrupo de transformações de $\glv V$ e seja $\mathcal T$ o grupo de translações. 
    Então o produto $\mathcal TG=\{T_tX\mid t\in V,\ X\in G\}$ é um subgrupo de $\Sym(V)$. 
\end{theorem}
\begin{proof}
    Seja $Y$ o conjunto de produtos no enunciado do teorema. Precisamos provar que $Y$ é fechado para 
    a composição e para tomar inversos. Sejam $T_{t_1}X_1$ e $T_{t_2}X_2$. Então temos que 
    \[
        (T_{t_1}X_{1})(T_{t_2}X_2)=T_{t_1}X_{1}T_{t_2}(X_1^{-1}X_1)X_2=
        T_{t_1}(X_{1}T_{t_2}X_1^{-1})X_1X_2=(T_{t_1}T_{X_1(t_2)})(X_1X_2)\in Y.
    \]  
    Além disso, temos que 
    \[
        (T_tX)^{-1}=X^{-1}T_t^{-1}=X^{-1}T_{-t}=X^{-1}T_{-t}XX^{-1}=T_{X^{-1}(-t)}X^{-1}\in Y.
    \]
\end{proof}
\section{Isometrias de $\R^n$}

Considere o espaço $\R^n$. Lembre que o produto escalar (ou produto interno) de dois vetores 
$v=(\alpha_1,\ldots,\alpha_n)$ e $w=(\beta_1,\ldots,\beta_n)$ é definido como 
\[
    v\cdot w=\alpha_1\beta_1+\cdots+\alpha_n\beta_n.
\]
O produto escalar pode ser escrita usando multiplicação matricial como 
\[
    v\cdot w=vw^t.
\]
Usando o produto escalar, podemos definir a norma $\|v\|$ de um vetor $v\in\R^n$ como
\[
    \|v\|=\sqrt{v\cdot v}.
\] 
A distância entre dois vetores $v,w\in \R^n$ pode ser definida como 
\[
    d(v,w)=\|v-w\|.
\]
Alêm disso, o cosseno do ângulo $\vartheta$ entre $v$ e $w$ é definido como 
\[
    \cos\vartheta=\frac{v\cdot w}{\|v\|\|w\|}
\]
Dois vetores $v,w\in \R^n$ são ortogonais se e somente se $v\cdot w=0$.

Da definição da norma fica clara que a norma está determinada pelo produto escalar. De acrodo do lema seguinte, 
a norma determina o produto escalar.

\begin{theorem}[Identidade de polarização]   Sejam $v,w\in \R^n$, então 
    \[
        v\cdot w=\frac 12\left(\|v+w\|-\|v\|-\|w\|\right).
\]
\end{theorem}
\begin{proof}
    Exercício.
\end{proof}

Uma transformação $T:\R^n\to\R^n$ que preserva distância (ou seja $d(T(v),T(w))=d(v,w)$ para 
todo $v,w\in \R^n$) chama-se \emph{isometria} de $\R^n$. Se $T$ é uma isometria e $T(v)=T(w)$, então 
\[
    0=d(T(v),T(w))=d(v,w);
\]
ou seja $v=w$. Isso implica que uma isometria é necessáriamente injetiva. Vamos ver que isometrias são 
também sobrejetivas, mas neste momenta esta afirmação não é tão fácil de provar. Por outro lado, se 
$T:\R^n\to\R^n$ é  uma isometria \emph{linear}, então ela é injetiva e precisa ser sobrejetiva. Logo, as 
isometrias lineares são invertíveis.

\begin{example}
    A translação $T_t:\R^n\to \R^n$ é uma isometria para todo $t\in\R^n$. De fato, temos 
    para $v,w\in\R^n$ que 
    \[
        d(v+t,w+t)=\|v+t-(w+t)\|=\|v-w\|=d(v,w).
    \]
\end{example}

\section{O grupo ortogonal}

\begin{theorem}
Seja $T:\R^n\to \R^n$ uma transformação linear. As seguintes são equivalentes para $T$.
\begin{enumerate}
    \item $T$ preserva o produto escalar; ou seja $T(v)\cdot T(w)=v\cdot w$ para todo $v,w\in\R^n$. 
    \item $T$ preserva a norma; ou seja $\|T(v)\|=\|v\|$ para todo $v\in\R^n$;
    \item $T$ preserva a distância $d(T(v),T(w))=d(v,w)$ para todo $v,w\in\R^n$.
\end{enumerate}
\end{theorem}
\begin{proof}
    O fato que (1) implica (2) e que (2) implica (3) segue das definições da norma e da distância. O fato que 
    (3) implica (1) segue dos fatos que $\|v\|=d(v,0)$, $T(0)=0$ ($T$ sendo linear)  e da identidade de polarização.
\end{proof}


Uma transformação linear $T:\R^n\to \R^n$ chama se \emph{ortogonal} se $T$ satisfaz uma (e então todas) das propriedades no teorema anterior. Por definição, as transformações ortogonais são exatamente as isometrias lineares do espaço $\R^n$.  Lembre que uma matriz $X$ é dita ortogonal se $X^tX=I$.

\begin{theorem}
    As seguintes afirmações são verdadeiras.
    \begin{enumerate}
        \item As transformações ortogonais formam um subgrupo de $\glv V$.
        \item Uma transformação $T:\R^n\to \R^n$ é ortogonal se e somente se sua matriz na base 
        canônica é ortogonal.
        \item O determinante de uma transformação ortogonal é $\pm 1$.  
    \end{enumerate}
\end{theorem}
\begin{proof}
    (1) Pode mostrar com uma conta direta que a composição de duas transformações ortogonais é ortogonal 
    e o inverso de uma transformação ortogonal é também ortogonal.

    (2) Se $T:\R^n\to \R^n$ é uma isometria linear, então $T$ preserva a norma de vetores e o ângulo entre vetores.
    Como os vetores $e_1,\ldots,e_n$ na base canônica formam um sistema ortonormal, 
    os vetores $T(e_1),\ldots,T(e_n)$ também formam um sistema ortonormal. Isso quer dizer que 
    $[T]_B^B$ é uma matriz ortogonal.
    
    Assuma agora que $T:\R^n\to \R^n$ é uma transformação linear tal que a sua matriz 
    $X$ na base canônica é ortogonal. Sejam $v,w\in\R^n$. Então  
    \[
        v\cdot w=vw^t=vX^tXw^t=(Xv^t)^t(Xw^t)=(Xv)\cdot (Xw)=T(v)\cdot T(w).
    \]
    Ou seja, $T$ preserva produto escalar e $T$ é uma isometria. 

    (3) Seja $T:\R^n\to \R^n$ uma isometria. Temos que $\det T=\det X$ onde $X$ é a matriz de $T$ na 
    base canônica. Como $X$ é uma matriz ortogonal, temos que 
    \[
        1=\det I=\det(X^tX)=\det(X^t)\det X=(\det X)^2
    \]
    e segue que $\det T=\det X=\pm 1$. 
\end{proof}

O grupo das transformações ortogonais de $\R^n$ é denotado por $O_n$. O subgrupo das transformações 
ortogonais com determinante $1$ é denotado por $SO_n$. Os grupos $O_n$ e $SO_n$ são chamados 
\emph{grupo ortogonal} e \emph{grupo especial ortogonal}. Os elementos de $SO_n$ são chamadas de \emph{rotações}
enquanto os demais elementos de $O_n$ são chamadas de \emph{reflexões}.







\end{document}

\documentclass[12pt]{amsart}
\usepackage{amsthm}
\usepackage{amssymb}
\usepackage{showkeys}
%\usepackage{tikz-cd}

\renewcommand{\a}{\mathfrak a}
\renewcommand{\b}{\mathfrak b}
\newcommand{\m}{\mathfrak m}
\newcommand{\n}{\mathfrak n}
\newcommand{\p}{\mathfrak p}
\newcommand{\q}{\mathfrak q}
\renewcommand{\r}{\mathfrak r}
\newcommand{\F}{\mathbb F}
\renewcommand{\L}{\mathbb L}
\newcommand{\Q}{\mathbb Q}
\newcommand{\N}{\mathbb N}
\newcommand{\Z}{\mathbb Z}
\newcommand{\C}{\mathbb C}


\newcommand{\K}{\mathbb K}
\newcommand{\R}{\mathbb R}
\newcommand{\fracf}[1]{\mbox{Frac}(#1)}
\newcommand{\spec}[1]{\mbox{Spec}(#1)}
\newcommand{\len}{\mbox{len}\,}
\newcommand{\id}{\mbox{id}\,}
\DeclareMathOperator{\sen}{\text{sen}}
%\newcommand{\sen}{\mbox{sen}\,}
\newcommand{\gl}[1]{\mbox{GL}_{#1}}
\newcommand{\SL}[1]{\mbox{SL}_{#1}}
\newcommand{\glv}[1]{\mbox{GL}(#1)}
\newcommand{\slv}[1]{\mbox{SL}(#1)}
\newcommand{\Sym}{\mbox{Sym}}
\usepackage[portuguese]{babel}
\newcommand{\rot}[1]{{\textrm{Rot}}_{#1}}
\newcommand{\refl}[1]{{\textrm{Ref}}_{#1}}




\newtheorem{theorem}{Teorema}
\newtheorem{corollary}[theorem]{Corolário}
\newtheorem{lemma}[theorem]{Lema}
\newtheorem{exercise}[theorem]{Exercício}

\theoremstyle{definition}
\newtheorem{example}[theorem]{Exemplo}
\newtheorem{definition}[theorem]{Definição}





\oddsidemargin 0pt
\evensidemargin 0pt
\textheight 8.1in \textwidth 6.3in


\relpenalty=10000
\binoppenalty=10000
\tolerance=500


\begin{document}

\title[Isometrias do espaço]{Transformações ortogonais em 2D e 3D}
\maketitle


\section{O plano (2D)}

\subsection{Realização matricial}
Lembre que a matriz da reflexão $R_t$ pelo eixo que tem ângulo $\alpha$ pelo eixo $x$ é
\[ 
\begin{pmatrix}\cos(2\alpha) & \sen(2\alpha) \\ \sen(2\alpha) & -\cos(2\alpha).
\end{pmatrix}
\]

Seja $T:\R^2\to\R^2$ uma transformação ortogonal e seja $e_1,e_2$ a base canônica. 
A matriz de $T$ tem os vetores $f_1=T(e_1)$ e $f_2=T(e_2)$ nas colunas. Pela ortogonalidade de 
$T$, $f_1$ e $f_2$ formam uma base ortonormal de $\R^2$. Assuma que $f_1=(a,b)$ com $\|f_1\|=a^2+b^2=1$. Então 
$f_2=(-b,a)$ ou $b_2=(b,-a)$.  Escolha um ângulo $\alpha$ tal  que $a=\cos\alpha$ e $b=\sen\alpha$. 
Então a matriz $[T]$ de $T$ tem duas possíveis formas:
\[ 
    \mbox{Caso I:}\ [T]=\begin{pmatrix}\cos \alpha & -\sen \alpha\\\sen\alpha &\cos\alpha\end{pmatrix}\\
    \qquad
    \mbox{Caso II:}\ [T]=\begin{pmatrix}\cos \alpha & \sen \alpha\\\sen\alpha &-\cos\alpha\end{pmatrix}.
\]

Seja 

\begin{lemma}
    No primeiro caso $T$ é a rotação $\rot\alpha$ pelo ângulo $\alpha$. No segundo caso, $T$ é a 
    reflexão $\refl{\alpha/2}$ pelo eixo que 
    tem ângulo $\alpha/2$ com o eixo $x$. 
\end{lemma}

\begin{lemma}
    Temos as seguintes regras para a composição de rotações e reflexões.
    \begin{enumerate}
        \item $\rot{\alpha}\circ\rot{\beta}=\rot{\alpha+\beta}$;
        \item $\refl{\alpha}\circ\refl{\beta}=\rot{2(\alpha-\beta)}$;
        \item $\rot{\alpha}\circ\refl{\beta}=\refl{\beta+\alpha/2}$;
        \item $\refl{\alpha}\circ\rot{\beta}=\refl{\alpha-\beta/2}$.
    \end{enumerate}
\end{lemma}
\begin{proof}
    (1) é exercício. Demonstremos (2). Seja $v\in\R^2$ e calculemos que 
    \begin{align*}
        &\refl{\alpha}\circ\refl{\beta}(v)=\begin{pmatrix}\cos(2\alpha) & \sen (2\alpha)\\\sen (2\alpha) &-\cos (2\alpha)\end{pmatrix}\cdot \begin{pmatrix}\cos (2\beta) & \sen (2\beta)\\\sen(2\beta) &-\cos(2\beta)\end{pmatrix}v\\
        &=\begin{pmatrix}
            \cos(2\alpha)\cos(2\beta)+\sen(2\alpha)\sen(2\beta) & \cos(2\alpha)\sen(2\beta)-\sen(2\alpha)\cos(2\beta)\\
            \sen(2\alpha)\cos(2\beta)-\cos(2\alpha)\sen(2\beta) & 
            \sen(2\alpha)\sen(2\beta)+\cos(2\alpha)\cos(2\beta)
        \end{pmatrix}v\\
        &=\begin{pmatrix} \cos(2(\alpha-\beta)) &  -\sen(2(\alpha-\beta))\\
            \sen(2(\alpha-\beta)) & \cos(2(\alpha-\beta)) 
        \end{pmatrix} v \\
        &=\rot{2(\alpha-\beta)}(v)
    \end{align*}
    
    (3) Temos que 
    \[
        \refl{\beta+\alpha/2}\circ(\refl{\beta})^{-1}=\refl{\beta+\alpha/2}\circ(\refl{\beta})=\rot{2(\beta-\alpha-\beta)}=\rot{\alpha}
    \]
    que implica afirmação (3). A demonstração de (4) é similar à demonstração de (3).
\end{proof}

\subsection{Realização com números complexos}

O vetor $(\alpha,\beta)\in\R^2$ pode ser identificado com o número complexo $\alpha+i\beta$. 
Cada número complexo $z$ pode ser escrito como $z=\|z\|(\cos \alpha+\sen\alpha)$ onde $\alpha$ é o ângulo 
(frequentamente chamado de argumento) que corresponde a $z$ e $\alpha\in[0,2\pi)$. Um número complexo 
$z$ com $\|z\|=1$ tem a forma $z=\cos\alpha+\sen\alpha$. Pela fórmula de Euler, 
\[
    \cos\alpha+i\sen\alpha = e^{i\alpha}=\exp(i\alpha)
\]

O conjugado de um número $z=\alpha+\beta i$ é $\bar z=\alpha-\beta i$. 

\begin{lemma}
    (1) Seja $z_\alpha=\cos\alpha+i\sen\alpha$. A aplicação 
    \[
        T:\C\to\C,\quad T(z)=z_\alpha\cdot z
    \] 
corresponde a rotação $\rot\alpha$ pelo ângulo $\alpha$ (em torno da origem). 

(2) Seja $T$ a reflexão pelo eixo que tem ângulo $\alpha$ com o eixo $x$. Então 
\[
    T(z)=z_{2\alpha}\cdot \bar z
\]
 \end{lemma}
 \begin{proof}
(1) Seja $z=\|z\|(\cos\beta+\sen\beta)$. Então 
\begin{align*}
    z_\alpha\cdot z&=\|z\|(\cos\alpha+\sen\alpha)(\cos\beta+\sen\beta)\\&=
    \cos\alpha\cos\beta-\cos\alpha\cos\beta+i(\cos\alpha\sen\beta+\sen\alpha\cos\beta)\\&=
    \|z\|(\cos(\alpha+\beta)+i(\cos\alpha+\beta)).
\end{align*}
 
(2) Claramente $\refl 0(z)=\bar z$. Então 
\[
    T(z)=z_{2\alpha}\cdot \bar z=\rot{2\alpha}\refl 0(z)=\refl{\alpha}(z).
\]
 \end{proof}

\end{document}

\documentclass[12pt]{amsart}
\usepackage{amsthm}
\usepackage{amssymb}
\usepackage{showkeys}
%\usepackage{tikz-cd}

\renewcommand{\a}{\mathfrak a}
\renewcommand{\b}{\mathfrak b}
\newcommand{\m}{\mathfrak m}
\newcommand{\n}{\mathfrak n}
\newcommand{\p}{\mathfrak p}
\newcommand{\q}{\mathfrak q}
\renewcommand{\r}{\mathfrak r}
\newcommand{\F}{\mathbb F}
\renewcommand{\L}{\mathbb L}
\newcommand{\Q}{\mathbb Q}
\newcommand{\N}{\mathbb N}
\newcommand{\Z}{\mathbb Z}
\newcommand{\C}{\mathbb C}
\newcommand{\Ha}{\mathbb H}


\newcommand{\K}{\mathbb K}
\newcommand{\R}{\mathbb R}
\newcommand{\fracf}[1]{\mbox{Frac}(#1)}
\newcommand{\spec}[1]{\mbox{Spec}(#1)}
\newcommand{\len}{\mbox{len}\,}
\newcommand{\id}{\mbox{id}\,}
\DeclareMathOperator{\sen}{\textrm{\rm sen}}
%\newcommand{\sen}{\mbox{sen}\,}
\newcommand{\gl}[1]{\mbox{GL}_{#1}}
\newcommand{\SL}[1]{\mbox{SL}_{#1}}
\newcommand{\glv}[1]{\mbox{GL}(#1)}
\newcommand{\slv}[1]{\mbox{SL}(#1)}
\newcommand{\Sym}{\mbox{Sym}}
\usepackage[portuguese]{babel}
\newcommand{\rot}[1]{{\textrm{Rot}}_{#1}}
\newcommand{\refl}[1]{{\textrm{Ref}}_{#1}}




\newtheorem{theorem}{Teorema}
\newtheorem{corollary}[theorem]{Corolário}
\newtheorem{lemma}[theorem]{Lema}
\newtheorem{exercise}[theorem]{Exercício}

\theoremstyle{definition}
\newtheorem{example}[theorem]{Exemplo}
\newtheorem{definition}[theorem]{Definição}





\oddsidemargin 0pt
\evensidemargin 0pt
\textheight 8.1in \textwidth 6.3in


\relpenalty=10000
\binoppenalty=10000
\tolerance=500


\begin{document}
\date{\today}
\title[Quatérnios]{Os quatérnios e as rotações em $\R^3$}
\maketitle


\section{A álgebra dos quatérnios}

Seja $\Ha$ o espaço vetorial de dimensão $4$ gerado por $1,i,j,k$. Introduzimos uma multiplicação em $\Ha$ com a seguinte tabela de multiplicação:

\begin{center}
        \begin{tabular}{|l||l|l|l|l|}
        \hline
            & 1   & $i$  & $j$  & $k$  \\ \hline\hline
        1   & 1   & $i$  & $j$  & $k$  \\ \hline
        $i$ & $i$ & $-1$ & $k$  & $-j$ \\ \hline
        $j$ & $j$ & $-k$ & $-1$ & $i$  \\ \hline
        $k$ & $k$ & $j$  & $-i$ & $-1$ \\ \hline
        \end{tabular}
\end{center}
Note que o conjunto $\{1,-1,i,-i,j,-j,k,-k\}$ é um grupo para esta multiplicação. 
A multiplicação entre os elementos $1,i,j,k$ será estendida com a regra distributiva. Um elemento de $\Ha$ chama-se um \emph{quatérnio} e e conjunto $\Ha$ chama-se a \emph{álgebra dos quatérnios}. A seguinte lema é fácil de verificar por conta direta.

\begin{lemma}
    A álgebra dos quatérnios é um espaço vetorial de dimensão $4$ com uma multiplicação bem definida. Além disso, a multiplicação é associativa, possui elemento neutro (o elemento $1$), mas não é comutativa. A estrutura satisfaz a lei distributiva:
    \[
        q_1(q_2+q_3)=q_1q_2+q_1q_3\quad\mbox{e}\quad(q_1+q_2)q_3=q_1q_3+q_2q_3.
    \]
\end{lemma}

Todo quatérnio $q\in\Ha$ pode ser escrito unicamente na forma $q=\alpha_q+v_q$ onde $\alpha_q\in\R$ e 
$v_q\in\left<i,j,k\right>$. Um quatérnio com $v_q=0$ chama-se \emph{escalar}, enquanto um quatérnio com 
$\alpha_q=0$ chama-se \emph{quatérnio puro}. Pode-se definir o produto escalar entre quatérnios como no espaço
$\R^3$ pela regra
\[
    (p,q)=\alpha_p\alpha_q+\beta_p\beta_q+\gamma_p\gamma_q+\delta_p\delta_q
\]
para todo $p=\alpha_p+\beta_pi+\gamma_pj+\delta_pk$ e $q=\alpha_q+\beta_qi+\gamma_qj+\delta_qk$.
(O produto escalar será denotado por $(\cdot,\cdot)$ para não confundir com a multiplicacão.)
Em relação com este produto escalar, os elementos $1$, $i$, $j$, e $k$ formam uma base ortonormal de 
$\Ha$.
A norma de um quatérnio na forma $q=\alpha+\beta i+\gamma j+\delta k$ é definida como 
\[
    \|q\|=\sqrt{(q,q)}=\sqrt{\alpha^2+\beta^2+\gamma^2+\delta^2},
\]
enquanto o conjugado $\bar q$ está definido como 
\[
    \bar q=\alpha-\beta i-\gamma j-\delta.
\]
A norma e o conjugado entre quatérnios satisfaz propriedades similares que a norma e o 
conjugado para números complexos.

\begin{lemma}
    As seguintes afirmações são verdadeiras para $q\in\Ha$.
    \begin{enumerate}
        \item $\alpha_q=(q+\bar q)/2$; 
        \item $v_q=(q-\bar q)/2$;
        \item $\overline{q_1+q_2}=\bar q_1+\bar q_2$ e $\overline{q_1q_2}=\bar q_1\cdot \bar q_2$;
        \item $\|q\|=0$ se e somente se $q=0$;
        \item $\|q_1+q_2\|\leq \|q_1\|+ \|q_2\|$;
        \item  $\|q_1\cdot q_2\|= \|q_1\|\|q_2\|$;
        \item $\|\alpha q\|=|\alpha|\|q\|$ para $\alpha\in\R$;
        \item $\|q\|^2=q\cdot \bar q$.
    \end{enumerate}
\end{lemma}
\begin{proof}
    Deixamos a maioria destas afirmações para exercício. 
    Para (8), calculemos que 
    \begin{align*}
        q\cdot \bar q&=(\alpha+\beta i+\gamma j+\delta k)(\alpha-\beta i-\gamma j-\delta k)\\&= 
        \alpha^2+\beta^2+\gamma^2+\delta^2=\|q\|^2.
    \end{align*}
\end{proof}

\begin{corollary}
    Seja $q=\alpha_q+v_q=\alpha+\beta i+\gamma j+\delta k\in\Ha\setminus\{0\}$. Então $q$ possui inverso multiplicativo 
    e 
    \[
        q^{-1}=\frac{\bar q}{\|q\|^2}=\frac{\alpha-\beta i-\gamma j-\delta k}{\alpha^2+\beta^2+\gamma^2+\delta^2}.
    \]
    Em outras palávras, $\Ha\setminus\{0\}$ é grupo para a multiplicação.
\end{corollary}
\begin{proof}
    Segue da afirmação (8) do lema anterior que 
    \[
        q\cdot \frac{\bar q}{\|q\|^2}=\frac{\|q\|^2}{\|q\|^2}=1.
    \]
\end{proof}

Um quatérnio $q\in\Ha$ chama-se \emph{unitário} se $\|u\|=1$.

\begin{lemma}
    O elemento $1\in\Ha$ é unitário, e se $q\in\Ha$ é unitário, então $q^{-1}$ é unitário. Logo, os quatérnios unitários formam um grupo para a multiplicação. Além disso, se $q\in\Ha$ é unitário, então $q^{-1}=\bar q$.   
\end{lemma}

Para $q_1,q_2\in\Ha$, defina o comutador 
\[
    [q_1,q_2]=\frac 12(q_1q_2-q_2q_1).
\]

\begin{lemma}
    O comutador satisfaz as seguintes propriedades para todo $q_1,q_2,q_3\in\Ha$:
    \begin{enumerate}
        \item $[q_1+q_2,q_3]=[q_1,q_3]+[q_2,q_3]$ e $[q_1,q_2+q_3]=[q_1,q_2]+[q_1,q_3]$ (distributividade);
        \item $[q_1,q_1]=0$ e $[q_1,q_2]=-[q_2,q_1]$ (anti-comutatividade);
        \item $[[q_1,q_2],q_3]+[[q_2,q_3],q_1]+[[q_3,q_1],q_2]=0$ (identidade de Jacobi).
    \end{enumerate}
\end{lemma}


As identidades no lema anterior implicam que a estrutura $(\Ha,+,[\cdot,\cdot])$ é uma \emph{álgebra de Lie}. 
Seja $Q$ o espaço dos quatérnios puros. Então $Q$ é um espaço vetorial de dimensão 3 gerado por $i$, $j$, e $k$. Note que $Q$ não é fechado para o produto $\cdot$ entre os quatérnios (por exemplo $i\cdot i=i^2=-1\not\in Q$), mas ele é fechado para o comutador. 
De fato, temos que $[i,j]=k$, $[j,k]=i$ e $[k,i]=j$. Ou seja, o comutador no espaço $k$ comporta-se 
exatamente como o produto vetorial $\times$ sobre $\R^3$. Além disso, $[1,q]=0$ para todo $q\in\Ha$.  


\begin{lemma}
    As seguintes propriedades são válidas para $q=\alpha_q+v_q$ e $p=\alpha_p+v_p$:
    \begin{enumerate}
        \item $[p,q]=[v_p,v_q]$;
        \item $p\cdot q=\alpha_p\alpha_q-(v_p,v_q)+\alpha_p v_q+\alpha_qv_p+v_p\times v_q=\alpha_p\alpha_q-(v_p,v_q)+\alpha_p v_q+\alpha_qv_p+[p,q]$;
        \item Se $p$ e $q$ são quatérnios puros, então $p\cdot q=-(p, q)+p\times q=-(p, q)+[p,q]$.
        \item Se $p$ e $q$ são quatérnios puros ortogonais, então $p\cdot q=p\times q$.
        \item Se $p$ é puro unitário, então $p^2=-1$.
    \end{enumerate}
\end{lemma}
\begin{proof}
    (1)--(4) Uma conta usando as definições. Para provar (5), note que item~(3) implica que 
    \[
        v^2=-(v, v)+v\times v=-(v, v)=-\|v\|^2=-1. 
    \]
\end{proof}


Se $q=\alpha_q+v_q\in\Ha$ com $v_q\neq 0$, então 
\[
    q=\alpha_q+\|v_q\|\frac{v_q}{\|v_q\|}=\alpha_q+\beta_qu_q
\]
onde $u_q$ é um quatérnio puro unitário. Além disso, se $\|q\|=1$, como $\alpha_q\perp v_q$,
\[
    1=\|q\|=\alpha_q^2+\beta_q^2
\]
então 
\[
    q=\cos\vartheta_q+\sen\vartheta_qu_q
\]
com algum ângulo $\vartheta\in[0,2\pi)$.

\begin{theorem}
    Todo quatérnio $q\in\Ha$ unitário pode ser escrito na forma 
    \[
        \cos\vartheta +(\sen\vartheta) u
    \]
    onde $u$ é um quatérnio puro unitário.
    Além disso, se $q\neq 1$, então esta expressão é única.
\end{theorem}
\begin{proof}
    If $v_q\neq 0$, então siga o processo antes do enunciado. Se $v_q=0$, então toma $\vartheta=0$ ou $\vartheta =\pi$ e $u$ arbitrário.
\end{proof}

\begin{lemma}
    Seja $u\in\Ha$ um quatérnio unitário. Então as aplicações 
    \[
        L_u:\Ha\to\Ha,\ L_u(q)=uq\quad\mbox{e}\quad R_u:\Ha\to\Ha,\ R_u(q)=qu^{-1}=q\bar u
    \]
    são transformações ortogonais de $\Ha$ com determinante~$1$; ou seja, $L_u$ e $R_u$ são 
    rotações de $\Ha\cong \R^4$. 
\end{lemma}
\begin{proof}
    Pela distributividade da multiplicação, temos que $L_u$ e $R_u$ são transformações lineares. Além disso
    \[
        \|L_u(q)\|=\|uq\|=\|u\|\|q\|=\|q\|
    \]
    e obtém-se similarmente que $\|R_u(q)\|=\|q\|$; 
    ou seja $L_u$ e $R_u$ preservam a norma. 
    Nós já provamos que para uma transformação linear isso é equivalente 
    a ser ortogonal. Precisamos ainda provar que $\det L_u=\det R_u=1$. Escreva 
    $u=\cos\vartheta+\sen\vartheta u_0$ onde $u_0$ é quatérnio puro unitário. 
    Neste caso $1\perp u_0$ e escolha um quatérnio puro unitário $v$ tal que 
    $u_0\perp v$ e seja $w=[u_0,v]=u_0\times v$. Então temos que a matriz de $L_u$ na base $1,u_0,v,w$ é 
    \[
        [L_{u_0}]=\begin{pmatrix} 0 & -1 & 0 & 0 \\ 1 & 0 & 0 & 0 \\ 0 & 0 & 0 & -1 \\ 0 & 0 & 1 & 0\end{pmatrix}
    \]
    e
    a matriz de $L_u=(\cos\vartheta)I+(\sen\vartheta) L_{u_0}$ é 
    \[
        [L_u]=\begin{pmatrix} \cos\vartheta & -\sen\vartheta & 0 & 0 \\ \sen\vartheta & \cos\vartheta & 0 & 0 \\ 0 & 0 & \cos\vartheta & -\sen\vartheta \\ 0 & 0 & \sen\vartheta & \cos\vartheta\end{pmatrix}.
    \]
    Segue que $L_u$ pode ser realizada como a composição de duas rotações: a primeira 
    no plano $\left<1,u\right>$ e a segunda no plano $\left<v,w\right>$ com ângulo $\vartheta$. 
    Temos que $\det L_u=(\cos^2\vartheta+\sen^2\vartheta)^2=1$. A computação para $R_u$ é similar. 
    Note que $u^{-1}=\bar u=\cos\vartheta-\sen\vartheta u_0$ e a matriz 
    de $R_u$ na mesma base será 
    \[
        [R_u]=\begin{pmatrix} \cos\vartheta & \sen\vartheta & 0 & 0 \\ -\sen\vartheta & \cos\vartheta & 0 & 0 \\ 0 & 0 & \cos\vartheta & -\sen\vartheta \\ 0 & 0 & \sen\vartheta & \cos\vartheta\end{pmatrix}
    \]
 Ou seja, $R_u$ faz uma rotação no plano $\left<1,u\right>$ com ângulo $-\vartheta$ e uma rotação no plano 
 $\left<v,w\right>$ por ângulo $\vartheta$.
\end{proof}

\begin{theorem}
    Seja $u=\cos\vartheta+(\sen\vartheta)u_0\in\Ha$ um quatérnio unitário. Defina 
    \[
        T_u:\Ha\to \Ha,\quad T_u(q)=uqu^{-1}=(L_u\circ R_u)(q).
    \]
    Então $T_u(1)=1$ e $T_u$ induz uma rotação do espaço $\R^3\cong\left<i,j,k\right>$. O eixo desta rotação é 
    $u_0$ e o seu ângulo é $2\vartheta$.
\end{theorem}
\begin{proof}
    Primeiro
    \[
        T_u(1)=u\cdot 1\cdot u^{-1}=u\cdot u^{-1}=1.
    \]
    Além disso, $T_u$ é uma composição de duas transformações ortogonais, e ela é ortigonal e temos ainda que 
    $\det T_u=\det L_u\cdot\det R_u=1$. Logo $T_u$ é uma rotação de $\Ha$. 
    Consequentemente, $T_u$ preserva $Q=\left<i,j,k\right>=\left<1\right>^{\perp}$. 
    Além disso, $T_u$ preserva a norma em $Q$ e assim a restrição de $T_u$ para $Q$ 
    é uma transformação ortogonal com determinante $1$. Portanto $T_u$ induz uma rotação 
    em $\left<i,j,k\right>$.  O eixo desta rotação pode ser 
    calculado por determinar um autovetor de $T_u$ em $\left<i,j,k\right>$ 
    que corresponde ao autovalor $1$. Mas note que 
    \[
        uu_0=(\cos\vartheta+(\sen\vartheta)u_0)u_0=u_0(\cos\vartheta+(\sen\vartheta)u_0)=
        (\cos\vartheta) u_0-\sen\vartheta=u_0u
    \]
    e assim 
    \[
        T_u(u_0)=uu_0u^{-1}=u_0uu^{-1}=u_0
    \]
    e obtemos que o eixo de $T_u$ em $\left<i,j,k\right>$ é $u$.

    Finalmente, temos que verificar a afirmação sobre o ângulo. Escreva $u=\cos\vartheta+(\sen\vartheta)u_0$
    onde $u_0$ é puro e unitário. Como na demonstração anterior, considere a base $1,u_0,v,w$ onde 
    $v\in Q$ unitário ortogonal a $u$ e $w=u\times v$. Como $T_u$ é a composição de $L_u$ e $R_u$, temos 
    que a matriz de $T_u$ nesta base é o produto das matrizes de $L_u$ e $R_u$ e assim
    \begin{align*}
        [T_u]&=\begin{pmatrix} \cos\vartheta & -\sen\vartheta & 0 & 0 \\ \sen\vartheta & \cos\vartheta & 0 & 0 \\ 0 & 0 & \cos\vartheta & -\sen\vartheta \\ 0 & 0 & \sen\vartheta & \cos\vartheta\end{pmatrix}
        \begin{pmatrix} \cos\vartheta & \sen\vartheta & 0 & 0 \\ -\sen\vartheta & \cos\vartheta & 0 & 0 \\ 0 & 0 & \cos\vartheta & -\sen\vartheta \\ 0 & 0 & \sen\vartheta & \cos\vartheta\end{pmatrix}\\
        &=\begin{pmatrix} 1 & 0 & 0 & 0 \\ 0 & 1 & 0 & 0 \\ 0 & 0 & \cos(2\vartheta) & -\sen(2\vartheta) \\ 0 & 0 & \sen(2\vartheta) & \cos(2\vartheta)\end{pmatrix}.
    \end{align*}
\end{proof}
\begin{corollary}
    Toda rotação $T$ de $R^3=\left<i,j,k\right>$ pode ser realizado como $T_u$ com algum $u\in\Ha$ unitário. 
\end{corollary}
\begin{proof}
    Seja $u_0$ o eixo de $T$ e $\vartheta$ o ângulo da rotação. Toma 
    \[
        u=\cos(\vartheta/2)+(\sen(\vartheta/2))u_0.
    \] 
    Pelo teorema anterior, $T=T_{u}$.
\end{proof}
\end{document}
 
\documentclass[12pt]{amsart}
\usepackage{amsthm}
\usepackage{amssymb}
\usepackage{showkeys}
%\usepackage{tikz-cd}

\renewcommand{\a}{\mathfrak a}
\renewcommand{\b}{\mathfrak b}
\newcommand{\m}{\mathfrak m}
\newcommand{\n}{\mathfrak n}
\newcommand{\p}{\mathfrak p}
\newcommand{\q}{\mathfrak q}
\renewcommand{\r}{\mathfrak r}
\newcommand{\F}{\mathbb F}
\renewcommand{\L}{\mathbb L}
\newcommand{\Q}{\mathbb Q}
\newcommand{\N}{\mathbb N}
\newcommand{\Z}{\mathbb Z}
\newcommand{\C}{\mathbb C}
\newcommand{\Ha}{\mathbb H}


\newcommand{\K}{\mathbb K}
\newcommand{\R}{\mathbb R}
\newcommand{\fracf}[1]{\mbox{Frac}(#1)}
\newcommand{\spec}[1]{\mbox{Spec}(#1)}
\newcommand{\len}{\mbox{len}\,}
\newcommand{\id}{\mbox{id}\,}
\DeclareMathOperator{\sen}{\textrm{\rm sen}}
%\newcommand{\sen}{\mbox{sen}\,}
\newcommand{\gl}[1]{\mbox{GL}_{#1}}
\newcommand{\SL}[1]{\mbox{SL}_{#1}}
\newcommand{\glv}[1]{\mbox{GL}(#1)}
\newcommand{\slv}[1]{\mbox{SL}(#1)}
\newcommand{\Sym}{\mbox{Sym}}
\usepackage[portuguese]{babel}
\newcommand{\rot}[1]{{\textrm{Rot}}_{#1}}
\newcommand{\refl}[1]{{\textrm{Ref}}_{#1}}




\newtheorem{theorem}{Teorema}
\newtheorem{corollary}[theorem]{Corolário}
\newtheorem{lemma}[theorem]{Lema}
\newtheorem{exercise}[theorem]{Exercício}

\theoremstyle{definition}
\newtheorem{example}[theorem]{Exemplo}
\newtheorem{definition}[theorem]{Definição}
\renewcommand{\P}{\mathcal P}
\renewcommand{\L}{\mathcal L}
\newcommand{\I}{\mathcal I}
\newcommand{\PP}{\mathbb P}
\newcommand{\agl}{\mbox{AGL}}





\oddsidemargin 0pt
\evensidemargin 0pt
\textheight 8.1in \textwidth 6.3in


\relpenalty=10000
\binoppenalty=10000
\tolerance=500


\begin{document}
\date{\today}
\title[Quatérnios]{Espaços afins e projetivos}
\maketitle

\section{Espaços afins e transformações afins}

Considere o espaço $\R^n$ mergulhado em $\R^{n+1}$ com a inclusão:
\[
    v=(\alpha_1,\ldots,\alpha_n)\mapsto \overline v=(\alpha_1,\ldots,\alpha_n,1)
\]
Lembre que o grupo $\agl_n$ é o grupo de transformações que podemos obter pela composição de uma 
transformação linear de $\R^n$ e uma translação em $\R^n$. 
Seja $L:\R^n\to \R^n$ uma transformação linear com matriz $X=[L]$ na base canônica. Seja $\bar X$ a 
matriz 
\[ 
    \bar X=\begin{pmatrix} X & \underline 0^t\\ \underline 0 & 1\end{pmatrix} 
\]
onde $\underline 0$ denota o vector nulo em $\R^n$. assum $\bar X$ é uma matriz 
$(n+1)\times(n+1)$. A matriz $\bar X$ chama-se \emph{matriz aumentada}.
É fácil verificar que $L(v)=w$ se e somente se $\bar X\bar v = \bar w$. 
\begin{example}
    Assuma que $L:\R^2\to\R^2$ é a rotação por $\pi/4$ (por volta da origem). 
    Então a sua matriz na base canônica é 
    \[
        X=\frac{\sqrt 2}{2}\begin{pmatrix} 1 & -1\\ 1 & 1\end{pmatrix}.
    \]
    A matriz aumentada que corresponde a $T$ é 
    \[
        \bar X=\begin{pmatrix} \sqrt 2/2 & -\sqrt 2/2 & 0 \\ \sqrt 2/2 & \sqrt 2/2 & 0 \\ 0 & 0 & 1
        \end{pmatrix}
    \]
\end{example}

Agora seja $b\in\R^n$ e defina a matriz 
\[
    X_b=\begin{pmatrix} I & b^t \\ \underline 0 & 1\end{pmatrix}.
\]
Considere $v\in\R^n$. Temos que 
\[
    X_b\bar v=\overline{v+b}=\overline{T_b(v)};
\]
ou seja, multiplicação por $X_b$ corresponde a translação pelo vetor $b$. 

\begin{example}
    Seja $b=(-1,2)\in\R^2$.
    Então 
    \[
    X_b=\begin{pmatrix} 1 & 0 & -1 \\ 0 & 1 & 2 \\ 0 & 0 & 1\end{pmatrix}.    
    \]
    Se $v=(\alpha,\beta)\in\R^2$, então $\bar v=(\alpha,\beta,1)$ e  
    \[
        X_b\bar v=
        X_b=\begin{pmatrix} 1 & 0 & -1 \\ 0 & 1 & 2 \\ 0 & 0 & 1\end{pmatrix}
        \cdot\begin{pmatrix} \alpha \\ \beta \\ 1 \end{pmatrix}=
        \begin{pmatrix}\alpha-1 \\ \beta +2  \\ 1 \end{pmatrix}=\overline{T_b(v)}.    
    \]
\end{example}

Finalmente, se $L:\R^n\to \R^n$ é uma transformação linear com matriz $X$ (na base canônica) 
e $b\in\R^n$, então defina 
\[
    X_{L,b}=\begin{pmatrix} X & b^t\\ \underline{0} & 1\end{pmatrix}.
\]
É fácil verificar que 
\[
    X_{L,b}\overline v=\overline{L(v)+b}.
\]
Ou seja, multiplicação pela matriz $X_{L,b}$ corresponde a composição $T_b\circ L$ em $\agl_n$.

\begin{example}
    Assuma que $b=(-1,2)$ e seja $L:\R^2\to\R^2$ a rotação por $\pi/4$ como no exemplo anterior. Então a 
    matriz
    \[
        X_{L,b}=\begin{pmatrix} \sqrt 2/2 & -\sqrt 2/2 & -1 \\ \sqrt 2/2 & \sqrt 2/2 & 2 \\ 0 & 0 & 1
        \end{pmatrix}
    \]
    Seja $v=(\alpha,\beta)\in\R^2$. Então $\bar v=(\alpha,\beta,1)$ e 
    \[
        \begin{pmatrix} \sqrt 2/2 & -\sqrt 2/2 & -1 \\ \sqrt 2/2 & \sqrt 2/2 & 2 \\ 0 & 0 & 1
        \end{pmatrix}\begin{pmatrix}\alpha \\ \beta\\ 1\end{pmatrix}=
        \overline{L(v)+b}.
    \]
\end{example}

Sejam $b_1,b_2\in \R^n$, $L_1,L_2\in \mbox{GL}_n$ e $v\in \R^n$. Então 
\[
    (T_{b_1}\circ L_1)\circ(T_{b_2}\circ L_2)v=
    L_1L_2v+L_1b_2+b_1=T_{L_1b_2+b_1}\circ (L_1\circ L_2).
\]
Pode verificar que 
\[
    X_{L_1,b_1}X_{L_2,b_2}=X_{L_1L_2,L_1b_2+b_1}.
\]

\begin{theorem}
    O grupo $\agl_n$ é isomorfo ao grupo de matrizes na forma 
    \[
        \{X_{L,b}\mid L\in\mbox{GL}_n\mbox{ e }b\in\R^n\}.
    \]
    O isomorfismo está dado por $T_b\circ L\mapsto X_{L,b}$. 
\end{theorem}



\section{Planos projetivos}

\begin{definition}
    A reta projetiva $\PP^1\F$ sobre um corpo $\F$ é o conjunto das retas em $\F^2$ que passam pela origem.
    Uma reta $L_{a,b}=\{(x,y)\in\F^2 \mid ax+by=0\}\subseteq \F^2$ é chamado de \emph{ponto} na reta projetiva $\PP^1\F$.
    Este ponto de $\PP^1\F$ pode ser representato com as coordenadas $[a,b]$ Estes coordenadas são 
    chamadas de coordenadas homgêneas. Note que $[a,b]$ representa um ponto em $\PP^1\F$ se e somente se 
    $(a,b)\neq (0,0)$ e $[\alpha a,\alpha b]$ representa a mesmo ponto que $[a,b]$ 
    para todo $\alpha\in\F\setminus\{0\}$. Assim, todo ponto de $\PP^1\F$ pode ser representado com as coordenadas 
    \[
        [1,b]\quad\mbox{ou}\quad [0,1]
    \]
    com algum $b\in \F$. O ponto $[0,1]$ é frequentamente chamado de \emph{ponto em infinito} e assim obtemos que $\PP^1\F$ pode ser identificado com $\F\cup\{P_\infty\}$ onde $P_\infty=[0,1]$ é o ponto em infinito.
\end{definition}

\begin{definition}
    Um plano projetivo $\Pi$ consiste de um conjunto $\P$ de pontos, um conjunto $\L$ de linhas (ou retas) e uma relação de incidência $\I\subseteq \P\times \I$ tal que 
    \begin{enumerate}
        \item Se $P_1,P_2\in \P$ distintos, então existe uma linha única linha $L \in \I$ tal que 
        $P_1\in L$, $P_2\in L$.
        \item Se $L_1,L_2\in \L$, então existe um único ponto $P\in \P$ tal que $P\in L_1$ e $P\in L_2$.
        \item Existem quatro pontos que nenhuma linha é incidente com mais que dois destes pontos.
    \end{enumerate}
\end{definition}

\begin{example}[Plano Euclediano Estendido]
    Considere o plano $\R^2$ com os pontos e linhas usuais. (Ou seja, os pontos são $P=(x,y)\in\R^2$ e as linhas 
    são conjuntos $\{(x,y)\mid ax+by=c\}$ com $(a,b,c)\neq (0,0,0)$.)
    Considere a relação de equivalência $\sim$ entre linhas 
    onde $L_1\sim L_2$ se e somente se $L_1$ e $L_2$ são paralelas. 
    Seja $[L]$ a classe de equivalência da linha $L$. 
    \begin{enumerate}
        \item Para cada classe $\ell=[L]$ introduza um novo ponto $P_\ell$ (ponto no infinito) e extenda 
    a incidência em tal modo que  $P_\ell\in L$ se e somente se $L\in \ell$.
    \item Introduza uma nova linha $L_\infty$ em tal modo que $L_\infty$ contem precisamente os pontos no infinito. A linha $L_\infty$ chama-se a linha em infinito.
    \end{enumerate}    
    A geometria obtida por este processo chama-se 
    \emph{Plano Euclediano Estendido} e é denotado por $E\R^2$. Deixamos para o leitor a verificação que $E\R^2$ é um plano projetivo.
\end{example}

\begin{example}
    Seja $\F$ um corpo qualquer (pode tomar por exemplo, $\F=\Q$, $\F=\R$, $\F=\C$, ou $\F=\F_p$), e considere 
    o espaço $\F^3$. Seja $\P$ o conjunto das retas que passam pela origem, e seja $\L$ o conjunto dos planos que passam pela origem. Um ponto $P$ é incidente com uma reta $L$, se $P\subseteq L$. É fácil verificar que 
    $\PP^3_\F=(\P,\L,\I)$ é um plano projetivo. Nós geralmente vamos considerar o plano $\PP^3=\PP^3_\R$. 
\end{example}



\end{document}
 


\end{document}