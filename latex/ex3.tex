\documentclass{amsart}


\usepackage{verbatim}
\usepackage{amssymb}

\usepackage[brazilian]{babel}
%\usepackage[latin1]{inputenc}
%\usepackage[T1]{fontenc}

\newcommand{\sym}[1]{{\mathsf S}_{#1}}
\newcommand{\alt}[1]{{\mathsf A}_{#1}}
\newcommand{\SL}[2]{{\sf SL}(#1,#2)}
\newcommand{\GL}[2]{{\sf GL}(#1,#2)}
\newcommand{\N}{\mathbb N}
\newcommand{\F}{\mathbb F}
\newcommand{\Z}{\mathbb Z}
\newcommand{\Q}{\mathbb Q}
\newcommand{\R}{\mathbb R}
\newcommand{\C}{\mathbb C}
\newcommand{\nullvec}{\overline 0}

\newcommand{\mat}[2]{{\mathsf M}_{#1,#2}}

\oddsidemargin 0pt
\evensidemargin 0pt
\textheight 8.1in \textwidth 6.3in


\relpenalty=10000
\binoppenalty=10000
\tolerance=500

\parskip 5pt
\parindent 0pt

\begin{document}


\begin{center}
\large Geometria com aplicações na gráfica computacional\\
{\bf\large Folha 3 de exercícios}\\
csaba@mat.ufmg.br
\end{center}


\bigskip

{\bf 1.} Descreva as possibilidades dos autovalores das transformações ortogonais $T:\R^4\to\R^4$. 
Dê exemplos com justificativas para todos os casos.

\medskip

{\bf 2.} Seja $X$ uma matriz $3\times 3$ representando uma reflexão de $\R^3$ em relação a um plano
$\Pi$
que passa pela origem. Seja $k=[k_x,k_y,k_z]$ um vetor normal unitário do plano.  Escreva 
as coordenadas $k_x,k_y,k_z$ em termos das entradas de $X$.

    \medskip

{\bf 3.} Seja $k\in\R^3$ um vetor unitário e seja $K$ a matriz da transformação linear 
$\R^3\to\R^3$ definido como $v\mapsto k\times v$. Mostre que a exponential 
\[
    \exp(\vartheta K)=\sum_{i=0}^\infty \frac{(\vartheta K)^i}{i!}
\]
é igual à matriz da rotação por ângulo $\vartheta$ no redor do eixo $k$.
\medskip


{\bf 4.} Uma transformação linear $T:\R^2\to\R^2$ chama-se \emph{cisalhamento} se a sua matriz na base canônica  está na forma 
\[
    \begin{pmatrix} 1 & \alpha \\ 0 & 1\end{pmatrix} \quad\mbox{ou}\quad 
    \begin{pmatrix} 1 & 0 \\ \alpha & 1\end{pmatrix}
\]
com algum $\alpha\in\R$. Uma transformação $T:\R^2\to\R^2$ chama-se escala se 
a sua matriz na base canônica está na forma 
\[
    \begin{pmatrix} \alpha & 0 \\ 0 &\beta\end{pmatrix}
\]
com $\alpha,\beta\in\R\setminus\{0\}$. 
\begin{enumerate}
    \item Mostre que uma rotação $\R^2\to\R^2$ pode ser decomposta para uma composição de três 
    cisalhamentos.
    \item Mostre que uma rotação $\R^2\to\R^2$ pode ser decomposta para uma composição de duas 
    cisalhamentos e uma escala.
\end{enumerate}

\medskip

{\bf 5.} Sejam $T:\R^3\to\R^3$ uma transformação ortogonal com $\det T=-1$. Mostre que 
\begin{enumerate}
    \item $T$ é uma reflexão em relação a um plano que passa pela origem; ou 
    \item $T$ é uma rotoreflexão (ou seja uma rotação por um eixo $k$ e ângulo $\vartheta$ 
    seguida por uma reflexão 
    em relação a um plano perpendicular a $k$).
\end{enumerate}

\medskip

{\bf 6} Seja $X$ a matriz de uma rotoreflexão em $\R^3$. Escreva o eixo $k$ e o ângulo $\vartheta$ 
em termos das entradas de $X$.
\end{document}

