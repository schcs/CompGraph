\documentclass[12pt]{amsart}
\usepackage{amsthm}
\usepackage{amssymb}
\usepackage{showkeys}
%\usepackage{tikz-cd}

\renewcommand{\a}{\mathfrak a}
\renewcommand{\b}{\mathfrak b}
\newcommand{\m}{\mathfrak m}
\newcommand{\n}{\mathfrak n}
\newcommand{\p}{\mathfrak p}
\newcommand{\q}{\mathfrak q}
\renewcommand{\r}{\mathfrak r}
\newcommand{\F}{\mathbb F}
\renewcommand{\L}{\mathbb L}
\newcommand{\Q}{\mathbb Q}
\newcommand{\N}{\mathbb N}
\newcommand{\Z}{\mathbb Z}
\newcommand{\C}{\mathbb C}
\newcommand{\Ha}{\mathbb H}


\newcommand{\K}{\mathbb K}
\newcommand{\R}{\mathbb R}
\newcommand{\fracf}[1]{\mbox{Frac}(#1)}
\newcommand{\spec}[1]{\mbox{Spec}(#1)}
\newcommand{\len}{\mbox{len}\,}
\newcommand{\id}{\mbox{id}\,}
\DeclareMathOperator{\sen}{\textrm{\rm sen}}
%\newcommand{\sen}{\mbox{sen}\,}
\newcommand{\gl}[1]{\mbox{GL}_{#1}}
\newcommand{\SL}[1]{\mbox{SL}_{#1}}
\newcommand{\glv}[1]{\mbox{GL}(#1)}
\newcommand{\slv}[1]{\mbox{SL}(#1)}
\newcommand{\Sym}{\mbox{Sym}}
\usepackage[portuguese]{babel}
\newcommand{\rot}[1]{{\textrm{Rot}}_{#1}}
\newcommand{\refl}[1]{{\textrm{Ref}}_{#1}}




\newtheorem{theorem}{Teorema}
\newtheorem{corollary}[theorem]{Corolário}
\newtheorem{lemma}[theorem]{Lema}
\newtheorem{exercise}[theorem]{Exercício}

\theoremstyle{definition}
\newtheorem{example}[theorem]{Exemplo}
\newtheorem{definition}[theorem]{Definição}
\renewcommand{\P}{\mathcal P}
\renewcommand{\L}{\mathcal L}
\newcommand{\I}{\mathcal I}
\newcommand{\PP}{\mathbb P}
\newcommand{\agl}{\mbox{AGL}}





\oddsidemargin 0pt
\evensidemargin 0pt
\textheight 8.1in \textwidth 6.3in


\relpenalty=10000
\binoppenalty=10000
\tolerance=500


\begin{document}
\date{\today}
\title{Espaços afins e projetivos}
\maketitle

\section{Espaços afins e transformações afins}

Considere o espaço $\R^n$ mergulhado em $\R^{n+1}$ com a inclusão:
\[
    v=(\alpha_1,\ldots,\alpha_n)\mapsto \overline v=(\alpha_1,\ldots,\alpha_n,1)
\]
Lembre que o grupo $\agl_n$ é o grupo de transformações que podemos obter pela composição de uma 
transformação linear de $\R^n$ e uma translação em $\R^n$. 
Seja $L:\R^n\to \R^n$ uma transformação linear com matriz $X=[L]$ na base canônica. Seja $\bar X$ a 
matriz 
\[ 
    \bar X=\begin{pmatrix} X & \underline 0^t\\ \underline 0 & 1\end{pmatrix} 
\]
onde $\underline 0$ denota o vector nulo em $\R^n$. assum $\bar X$ é uma matriz 
$(n+1)\times(n+1)$. A matriz $\bar X$ chama-se \emph{matriz aumentada}.
É fácil verificar que $L(v)=w$ se e somente se $\bar X\bar v = \bar w$. 
\begin{example}
    Assuma que $L:\R^2\to\R^2$ é a rotação por $\pi/4$ (por volta da origem). 
    Então a sua matriz na base canônica é 
    \[
        X=\frac{\sqrt 2}{2}\begin{pmatrix} 1 & -1\\ 1 & 1\end{pmatrix}.
    \]
    A matriz aumentada que corresponde a $T$ é 
    \[
        \bar X=\begin{pmatrix} \sqrt 2/2 & -\sqrt 2/2 & 0 \\ \sqrt 2/2 & \sqrt 2/2 & 0 \\ 0 & 0 & 1
        \end{pmatrix}
    \]
\end{example}

Agora seja $b\in\R^n$ e defina a matriz 
\[
    X_b=\begin{pmatrix} I & b^t \\ \underline 0 & 1\end{pmatrix}.
\]
Considere $v\in\R^n$. Temos que 
\[
    X_b\bar v=\overline{v+b}=\overline{T_b(v)};
\]
ou seja, multiplicação por $X_b$ corresponde a translação pelo vetor $b$. 

\begin{example}
    Seja $b=(-1,2)\in\R^2$.
    Então 
    \[
    X_b=\begin{pmatrix} 1 & 0 & -1 \\ 0 & 1 & 2 \\ 0 & 0 & 1\end{pmatrix}.    
    \]
    Se $v=(\alpha,\beta)\in\R^2$, então $\bar v=(\alpha,\beta,1)$ e  
    \[
        X_b\bar v=
        X_b=\begin{pmatrix} 1 & 0 & -1 \\ 0 & 1 & 2 \\ 0 & 0 & 1\end{pmatrix}
        \cdot\begin{pmatrix} \alpha \\ \beta \\ 1 \end{pmatrix}=
        \begin{pmatrix}\alpha-1 \\ \beta +2  \\ 1 \end{pmatrix}=\overline{T_b(v)}.    
    \]
\end{example}

Finalmente, se $L:\R^n\to \R^n$ é uma transformação linear com matriz $X$ (na base canônica) 
e $b\in\R^n$, então defina 
\[
    X_{L,b}=\begin{pmatrix} X & b^t\\ \underline{0} & 1\end{pmatrix}.
\]
É fácil verificar que 
\[
    X_{L,b}\overline v=\overline{L(v)+b}.
\]
Ou seja, multiplicação pela matriz $X_{L,b}$ corresponde a composição $T_b\circ L$ em $\agl_n$.

\begin{example}
    Assuma que $b=(-1,2)$ e seja $L:\R^2\to\R^2$ a rotação por $\pi/4$ como no exemplo anterior. Então a 
    matriz
    \[
        X_{L,b}=\begin{pmatrix} \sqrt 2/2 & -\sqrt 2/2 & -1 \\ \sqrt 2/2 & \sqrt 2/2 & 2 \\ 0 & 0 & 1
        \end{pmatrix}
    \]
    Seja $v=(\alpha,\beta)\in\R^2$. Então $\bar v=(\alpha,\beta,1)$ e 
    \[
        \begin{pmatrix} \sqrt 2/2 & -\sqrt 2/2 & -1 \\ \sqrt 2/2 & \sqrt 2/2 & 2 \\ 0 & 0 & 1
        \end{pmatrix}\begin{pmatrix}\alpha \\ \beta\\ 1\end{pmatrix}=
        \overline{L(v)+b}.
    \]
\end{example}

Sejam $b_1,b_2\in \R^n$, $L_1,L_2\in \mbox{GL}_n$ e $v\in \R^n$. Então 
\[
    (T_{b_1}\circ L_1)\circ(T_{b_2}\circ L_2)v=
    L_1L_2v+L_1b_2+b_1=T_{L_1b_2+b_1}\circ (L_1\circ L_2).
\]
Pode verificar que 
\[
    X_{L_1,b_1}X_{L_2,b_2}=X_{L_1L_2,L_1b_2+b_1}.
\]

\begin{theorem}
    O grupo $\agl_n$ é isomorfo ao grupo de matrizes na forma 
    \[
        \{X_{L,b}\mid L\in\mbox{GL}_n\mbox{ e }b\in\R^n\}.
    \]
    O isomorfismo está dado por $T_b\circ L\mapsto X_{L,b}$. 
\end{theorem}



\section{Planos projetivos}

\begin{definition}
    A reta projetiva $\PP^1\F$ sobre um corpo $\F$ é o conjunto das retas em $\F^2$ que passam pela origem.
    Uma reta $L_{a,b}=\{(x,y)\in\F^2 \mid ax+by=0\}\subseteq \F^2$ é chamado de \emph{ponto} na reta projetiva $\PP^1\F$.
    Este ponto de $\PP^1\F$ pode ser representato com as coordenadas $[a,b]$ Estes coordenadas são 
    chamadas de coordenadas homgêneas. Note que $[a,b]$ representa um ponto em $\PP^1\F$ se e somente se 
    $(a,b)\neq (0,0)$ e $[\alpha a,\alpha b]$ representa a mesmo ponto que $[a,b]$ 
    para todo $\alpha\in\F\setminus\{0\}$. Assim, todo ponto de $\PP^1\F$ pode ser representado com as coordenadas 
    \[
        [1,b]\quad\mbox{ou}\quad [0,1]
    \]
    com algum $b\in \F$. O ponto $[0,1]$ é frequentamente chamado de \emph{ponto em infinito} e assim obtemos que $\PP^1\F$ pode ser identificado com $\F\cup\{P_\infty\}$ onde $P_\infty=[0,1]$ é o ponto em infinito.
\end{definition}

\begin{definition}
    Um plano projetivo $\Pi$ consiste de um conjunto $\P$ de pontos, um conjunto $\L$ de linhas (ou retas) e uma relação de incidência $\I\subseteq \P\times \I$ tal que 
    \begin{enumerate}
        \item Se $P_1,P_2\in \P$ distintos, então existe uma linha única linha $L \in \I$ tal que 
        $P_1\in L$, $P_2\in L$.
        \item Se $L_1,L_2\in \L$, então existe um único ponto $P\in \P$ tal que $P\in L_1$ e $P\in L_2$.
        \item Existem quatro pontos que nenhuma linha é incidente com mais que dois destes pontos.
    \end{enumerate}
\end{definition}

\begin{example}[Plano Euclediano Estendido]
    Considere o plano $\R^2$ com os pontos e linhas usuais. (Ou seja, os pontos são $P=(x,y)\in\R^2$ e as linhas 
    são conjuntos $\{(x,y)\mid ax+by=c\}$ com $(a,b,c)\neq (0,0,0)$.)
    Considere a relação de equivalência $\sim$ entre linhas 
    onde $L_1\sim L_2$ se e somente se $L_1$ e $L_2$ são paralelas. 
    Seja $[L]$ a classe de equivalência da linha $L$. 
    \begin{enumerate}
        \item Para cada classe $\ell=[L]$ introduza um novo ponto $P_\ell$ (ponto no infinito) e extenda 
    a incidência em tal modo que  $P_\ell\in L$ se e somente se $L\in \ell$.
    \item Introduza uma nova linha $L_\infty$ em tal modo que $L_\infty$ contem precisamente os pontos no infinito. A linha $L_\infty$ chama-se a linha em infinito.
    \end{enumerate}    
    A geometria obtida por este processo chama-se 
    \emph{Plano Euclediano Estendido} e é denotado por $E\R^2$. Deixamos para o leitor a verificação que $E\R^2$ é um plano projetivo.
\end{example}

\begin{example}
    Seja $\F$ um corpo qualquer (pode tomar por exemplo, $\F=\Q$, $\F=\R$, $\F=\C$, ou $\F=\F_p$), e considere 
    o espaço $\F^3$. Seja $\P$ o conjunto das retas que passam pela origem, e seja $\L$ o conjunto dos planos que passam pela origem. Um ponto $P$ é incidente com uma reta $L$, se $P\subseteq L$. É fácil verificar que 
    $\PP^3_\F=(\P,\L,\I)$ é um plano projetivo. Nós geralmente vamos considerar o plano $\PP^3=\PP^3_\R$. 
\end{example}

\subsection{Coordenadas homogêneas}
Considere o plano $E\R^2$ estendido. Introduzimos coordenadas homogêneas para pontos e retas.
\begin{enumerate}
\item Seja $p=(x,y)\in\R^2$. A tripla $[\lambda x,\lambda y,\lambda]$ 
é coordenada homogênea para $p$ com $\lambda\in\R\setminus\{0\}$. 
\item Seja $p$ um ponto em infinito que corresponde a uma classe paralela de linhas $ax+by+c=0$ com $a,b\in\R$ 
fixados. A tripla $[\lambda a,\lambda b,0]$ é coordenada homogênea de $p$ com qualquer 
$\lambda\in\R\setminus\{0\}$. 
\item Seja $\ell$ uma reta a equação $ax+by+c=0$. Então a tripla $[a,b,c]$ é coordenada homegênea para $\ell$. 
\item Seja $\ell$ a reta no infnito. Então $[0,0,\lambda]$ é coordenada homogênea de $\ell$ para qualquer 
$\lambda\in\R\setminus\{0\}$. 
\end{enumerate}

\begin{lemma}
    Todo ponto e toda reta em $E\R^2$ possui coordenadas homogêneas. Além disso 
    $[\alpha_1,\alpha_2,\alpha_3]$ e $[\beta_1,\beta_2,\beta_3]$ representam o mesmo ponto/reta se e somente 
    se existe $\lambda\in\R\setminus\{0\}$ tal que $\beta_i=\lambda\alpha_i$ para todo $i\in\{1,2,3\}$.  
\end{lemma}
\begin{proof}
    Segue as definições.
\end{proof}

\begin{lemma}
    Seja $p$ um ponto e $\ell$ uma reta representados pelas coordenadas $[a,b,c]$ e $[u,v,w]$. Temos que 
    $p\in\ell$ se e somente se $[a,b,c]\cdot [u,v,w]=0$. (produto escalar)
\end{lemma}
\begin{proof}
    Segue as definições.
\end{proof}

\begin{lemma}
    Assuma que $[u_1,v_1,w_1]$ e $[u_2,v_2,w_2]$ são retas distintas em $E\R^2$. Temos que 
    as coordenadas homegeneas do único ponto na interseção das duas retas são 
    dadas pelo produto vetorial
    $[u_1,v_1,w_1]\times [u_2,v_2,w_2]$. 
\end{lemma}
\begin{proof}
    Note que o produto misto 
    \[
        [u_1,v_1,w_1]\cdot([u_1,v_1,w_1]\times [u_2,v_2,w_2])=\det\begin{pmatrix}
            u_1 & u_1 & u_2\\ v_1 & v_1 & v_2\\w_1 & w_1 & w_2\end{pmatrix}=0. 
    \]
    Uma conta similar mostra que $[u_2,v_2,w_2]\cdot([u_1,v_1,w_1]\times [u_2,v_2,w_2])=0$. 
    Então o ponto com coordenadas homogêneas $[u_1,v_1,w_1]\times [u_2,v_2,w_2]$ está nas duas linhas. 
    Pelos axiomas do plano projetivo, este é o único ponto nas duas retas. 

    Note que $[u_1,v_1,w_1]\times [u_2,v_2,w_2]\neq [0,0,0]$ são retas distintas, e assim 
    $[u_1,v_1,w_1]$ e $[u_2,v_2,w_2]\neq [0,0,0]$. 
\end{proof}

\begin{lemma}
    Assuma que $[a_1,b_1,c_1]$ e $[a_2,b_2,c_2]$ são pontos distintos em $E\R^2$. Temos que 
    as coordenadas homegeneas da única reta que passa por estes dois pontos  são 
    dadas pelo produto vetorial
    $[a_1,b_1,c_1]\times [a_2,b_2,c_2]$. 
\end{lemma}
\begin{proof}
    Igual ao lema anterior.
\end{proof}

\begin{lemma}
    Assuma que $p_1=[a_1,b_1,c_1]$, $p_2=[a_2,b_2,c_2]$ e $p_3=[a_3,b_3,c_3]$ são pontos em $E\R^2$. 
    Os pontos $p_1,p_2,p_3$ são collineares se e somente se
    \[
        \det\begin{pmatrix} a_1 & a_2 & a_3\\ b_1 & b_2 & b_3\\ c_1 & c_2 & c_3\end{pmatrix}=0.
    \]
\end{lemma}
\begin{proof}
    Se $p_2=p_3$, então os três pontos são collineares e o determinante no teorema é também igual a zero. 
    Assuma que $p_2\neq p_3$.
    Os pontos $p_1$, $p_2$, e $p_3$ são collineares se e somente se o ponto $p_1$ está na reta determinada 
    por $p_2$ e $p_3$. Isso occorre se e somente se 
    \[
        0=p_1\cdot (p_2\times p_3)=\det\begin{pmatrix} a_1 & a_2 & a_3\\ b_1 & b_2 & b_3\\ c_1 & c_2 & c_3\end{pmatrix}.
    \]
\end{proof}

\begin{lemma}
    Assuma que $\ell_1=[u_1,v_1,w_1]$, $\ell_2=[u_2,v_2,w_2]$ e $\ell_3=[u_3,v_3,w_3]$ são retas em $E\R^2$. 
    As retas $\ell_1,\ell_2,\ell_3$ são concorrentes se e somente se
    \[
        \det\begin{pmatrix} u_1 & u_2 & u_3\\ v_1 & v_2 & v_3\\ w_1 & w_2 & w_3\end{pmatrix}=0.
    \]
\end{lemma}

\subsection{Espaço estendido $E\R^3$}
Considere $\R^3$. Para cada classe $C$ de retas paralelas, introduza um ponto em infinito $P_C$ 
e $P_C \in\ell$ para todo $\ell \in C$. Para classe $D$ de planos paralelos, introduza uma reta $\ell_D$ 
em infinito tal que $\ell_D\in \Pi$ para todo $\Pi\in D$ e $\ell_D$ contém os pontos em infinito que 
são contidos nas retas de $\Pi$. Finalmente, introduza um plano $\Pi_\infty$ em infinito que contém 
as retas em infinito.

As coordenadas homogêneas dos pontos $(a,b,c)\in\R^3$ são $[\lambda a,\lambda b,\lambda c,\lambda]$ 
com $\lambda\in\R\setminus\{0\}$. Se $C$ é uma classe de retas paralelas, paralelo ao vetor $(v_1,v_2,v_3)$, 
então as coordenadas homogêneas do ponto correspondente são $[v_1,v_2,v_3,0]$. 
Um plano definida pela equação $ax+by+cz+d=0$ tem coordenadas homogêneas $[a,b,c,d]$, enquanto o plano em 
infinito tem coordenadas homogêneas $[0,0,0,\lambda]$ com $\lambda\in\R\setminus\{0\}$. 

\begin{lemma}
    Sejam $p_1=[a_1,b_1,c_1,d_1]$, $p_2=[a_2,b_2,c_2,d_2]$, $p_3=[a_2,b_2,c_2,d_2]$ pontos em $E\R^3$.
    Os pontos $p_1,p_2,p_3$ são collineares se e somente se os vetores são linearmente dependentes.   
\end{lemma}
\begin{proof}
    Exercício.
\end{proof}

Se $p_1=[a_1,b_1,c_1,d_1]$, $p_2=[a_2,b_2,c_2,d_2]$ são pontos distintos em $E\R^3$, então os pontos 
da reta determinada por $p_1$ e $p_2$ são os pontos com coordenadas $[a,b,c,d]$ tal que 
$[a,b,c,d]$ é uma combinação linear de $[a_1,b_1,c_1,d_1]$ e $[a_2,b_2,c_2,d_2]$. Assim a reta que passa
pelos pontos $p_1$ e $p_2$ pode ser representada pela seguinte representação paramétrica:
\[
    \alpha[a_1,b_1,c_1,d_1]+\beta[a_2,b_2,c_2,d_2]\quad\mbox{onde}\quad \alpha,\beta\in\R
    \quad\mbox{com}\quad (\alpha,\beta)\neq (0,0).
\]

\subsection{Coordenadas Pl\"ucker}
Seja $p_1=[x_1,x_2,x_3,x_4]$ e $p_2=[y_1,y_2,y_3,y_4]$ dois pontos em $E\R^3$. Para $i,j\in\{1,\ldots,4\}$,  
ponha 
\[
    p_{i,j}=\det\begin{pmatrix} x_i & y_i\\ x_j & y_j\end{pmatrix}
\]
As coordenadas Plucker da reta que passa pelos pontos $p_1$ e $p_2$ são definidas como 
\[
    [p_{1,2},p_{1,3},p_{1,4},p_{3,4},p_{4,2},p_{2,3}].
\]
\end{document}
 