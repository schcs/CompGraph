\documentclass{amsart}


\usepackage{verbatim}
\usepackage{amssymb}

\usepackage[brazilian]{babel}
%\usepackage[latin1]{inputenc}
%\usepackage[T1]{fontenc}

\newcommand{\sym}[1]{{\mathsf S}_{#1}}
\newcommand{\alt}[1]{{\mathsf A}_{#1}}
\newcommand{\SL}[2]{{\sf SL}(#1,#2)}
\newcommand{\GL}[2]{{\sf GL}(#1,#2)}
\newcommand{\N}{\mathbb N}
\newcommand{\F}{\mathbb F}
\newcommand{\Z}{\mathbb Z}
\newcommand{\Q}{\mathbb Q}
\newcommand{\R}{\mathbb R}
\newcommand{\C}{\mathbb C}
\newcommand{\nullvec}{\overline 0}
\DeclareMathOperator{\sen}{\textrm{\rm sen}}

\newcommand{\mat}[2]{{\mathsf M}_{#1,#2}}

\oddsidemargin 0pt
\evensidemargin 0pt
\textheight 8.1in \textwidth 6.3in


\relpenalty=10000
\binoppenalty=10000
\tolerance=500

\parskip 5pt
\parindent 0pt

\begin{document}


\begin{center}
\large Geometria com aplicações na gráfica computacional\\
{\bf\large Folha 5 de exercícios}\\
csaba@mat.ufmg.br
\end{center}


\bigskip

{\bf 1.} Seja $p$ um quatérnio unitário e $t\in\R$. Mostre que $(qp^t\bar q)=(qp\bar q)^t$ onde $\bar q$ é 
o conjugado de $q$. 

\medskip

{\bf 2.} Mostre, para um quatérnio unitário $q$ e $a,b\in\R$ que 
\[
    q^aq^b=q^{a+b}\quad\mbox{e}\quad (q^a)^b=q^{ab}.
\]
\medskip

{\bf 3.} Verifique as seguintes igualdades para quatérnios unitários $p$ e $q$:
\begin{enumerate}
    \item $\mbox{slerp}(p,q,t)=p(\bar pq)^t$;
    \item $\mbox{slerp}(p,q,t)=(p\bar q)^{1-t}q$;
    \item $\mbox{slerp}(p,q,t)=(q\bar p)^{t}p$;
    \item $\mbox{slerp}(p,q,t)=q(\bar qp)^{1-t}q$.
\end{enumerate}
Deduza que $\mbox{slerp}(p,q,t)=\mbox{slerp}(q,p,1-t)$. 

\medskip


{\bf 4.} Mostre que a multiplicação entre dois quatérnios pode ser efetuada por apanas 8 multiplicações
entre números reais. [Dica: Consulte Exercício~20 na página 112 do livro "Fundamentos da Computação Gráfica" por Gomes e Velho.]

\medskip

{\bf 5.} Assuma que $r$ é um quatérnio puro e unitário. Considere o mapa $R:\mathbb H\to\mathbb H$
definido por $q\mapsto rqr$. 
\begin{enumerate}
    \item Mostre que $\left<i,j,k\right>$ é um subespaço $R$-invariante.
    \item Mostre que a restrição de $R$ para o suespaço $\R^3=\left<i,j,k\right>$
    pode ser vista como a reflexão de $\R^3$ em relação ao plano perpendicular a $r$. 
\end{enumerate}


\medskip

{\bf 6.} Uma matriz quadrada $A$ com entradas complexas e dita unitária se $A^*A=I$ onde $A^*$ é a conjugada transposta de $A$ (ou seja, $A^*$ é obtida de $A$ por tomar a conjugada complexa de cada entrada e depois 
tomar a transposta.)  Denote por $SU_2$ o grupo de matrizes unitárias $2\times 2$ com determinante $1$. 
Para cada quatérnio unitário $q=a+bi+cj+dk$, denote por $A_q$ a matriz 
\[
\begin{pmatrix} a+ib & c+id \\ -c+id & a-ib\end{pmatrix}.
\]
\begin{enumerate}
\item Verifique que $A_q\in SU_2$.
    \item   Verifique que $\psi$ é um homomorfismo de grupos, ou seja $\psi(q_1q_2)=\psi(q_1)\psi(q_2)$. 
    \item Demonstre que o mapa $\psi: q\mapsto A_q$ é uma bijeção entre o grupo de quatérnios unitários e 
    $SU_2$.
\end{enumerate}

\end{document}

