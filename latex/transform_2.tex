\documentclass[12pt]{amsart}
\usepackage{amsthm}
\usepackage{amssymb}
\usepackage{showkeys}
%\usepackage{tikz-cd}

\renewcommand{\a}{\mathfrak a}
\renewcommand{\b}{\mathfrak b}
\newcommand{\m}{\mathfrak m}
\newcommand{\n}{\mathfrak n}
\newcommand{\p}{\mathfrak p}
\newcommand{\q}{\mathfrak q}
\renewcommand{\r}{\mathfrak r}
\newcommand{\F}{\mathbb F}
\renewcommand{\L}{\mathbb L}
\newcommand{\Q}{\mathbb Q}
\newcommand{\N}{\mathbb N}
\newcommand{\Z}{\mathbb Z}
\newcommand{\C}{\mathbb C}


\newcommand{\K}{\mathbb K}
\newcommand{\R}{\mathbb R}
\newcommand{\fracf}[1]{\mbox{Frac}(#1)}
\newcommand{\spec}[1]{\mbox{Spec}(#1)}
\newcommand{\len}{\mbox{len}\,}
\newcommand{\id}{\mbox{id}\,}
\newcommand{\sen}{\mbox{sen}\,}
\newcommand{\gl}[1]{\mbox{GL}_{#1}}
\newcommand{\SL}[1]{\mbox{SL}_{#1}}
\newcommand{\glv}[1]{\mbox{GL}(#1)}
\newcommand{\slv}[1]{\mbox{SL}(#1)}
\newcommand{\Sym}{\mbox{Sym}}
\usepackage[portuguese]{babel}

\newtheorem{theorem}{Teorema}
\newtheorem{corollary}[theorem]{Corolário}
\newtheorem{lemma}[theorem]{Lema}
\newtheorem{exercise}[theorem]{Exercício}

\theoremstyle{definition}
\newtheorem{example}[theorem]{Exemplo}
\newtheorem{definition}[theorem]{Definição}




\oddsidemargin 0pt
\evensidemargin 0pt
\textheight 8.1in \textwidth 6.3in


\relpenalty=10000
\binoppenalty=10000
\tolerance=500


\begin{document}

\title[Isometrias do espaço]{Isometrias de $\R^n$}
\maketitle

\section{Transformações}

Seja $\Omega$ um conjunto. Uma aplicação $f:\Omega\to \Omega$ chama-se uma \emph{transformação} de $\Omega$. A transformação $f$ é dita \emph{injetiva} se $f(v)=f(w)$ implica $v=w$ para todo $v,w\in\Omega$;
$f$ chama-se \emph{sobrejetiva} se para todo $w\in\Omega$ existe $v\in\Omega$ tal que $f(v)=w$. A transformação
$f$ chama-se \emph{bijetiva} ou \emph{invertível} se ela é injetiva e sobrejetiva. Se $f$ é uma transformação 
invertível, então existe a sua inversa $f^{-1}:\Omega\to\Omega$ definida pela regra que 
$f(v)=w$ se e somente se $f^{-1}(w)=v$ para todo $v,w\in\Omega$.  

As transformações de $\Omega$ podem ser compostas. Se $f,g:\Omega\to \Omega$ então $f\circ g:\Omega\to \Omega$ 
é definida como $(f\circ g)(v)=f(g(v))$ para todo $v\in \Omega$. A composição de transformações é associativa no sentido que $(f\circ g)\circ h=f\circ(g\circ h)$ para todo $f,g,h:\Omega\to \Omega$. 

Nós vamos estudar principalmente as transformações do plano $\R^2$ e o espaço $\R^3$. 

\begin{example}
    Todo conjunto $\Omega$ tem a transformação identidade $\id_\Omega:\Omega\to \Omega$, $v\mapsto v$ 
    para todo $v\in\Omega$. Se $f:\Omega\to\Omega$ é uma transformação invertível, então 
    $f\circ f^{-1}=f^{-1}\circ f=\id_\Omega$. 
\end{example}

\begin{example}
    Seja $\Omega=\R^n$ com $n\geq 1$. Uma transformação $T:\R^n\to\R^n$ é dita linear se 
    \[
        T(\alpha v+\beta w)=\alpha T(v)+\beta T(w)
    \]
    para todo $v,w\in\R^n$ e $\alpha,\beta\in\R$. Transformações lineares são estudadas em álgebra linear.
    Uma transformação linear $T:\R^n\to \R^n$ é invertível se e somente se 
    \[
        \ker T=\{v\in \R^n\mid T(v)=0\}=\{0\}.
    \]
\end{example}

\begin{example}
    Seja $t\in\R^n$ e considere a transformação $T_t:\R^n\to \R^n$ definido como 
    \[
        T_t(v)=v+t.
    \]
    A transformação $T_t$ é chamado \emph{a translação de $\R^n$} pelo vetor $t$. Note que se 
    $t\neq 0$, então $T_t$ não é linear, pois $T_t(0)=t\neq 0$. A transformação $T_t$ é invertível e 
    $T_t^{-1}=T_{-t}$. 
\end{example}

\section{Grupos}
Seja $G$ um conjunto não vazio com uma operação que pode ser denotada por $\cdot$ (ou por $+$, ou simplesmente 
por concatenação). Isso quer dizer que com cada par de elementos 
$a,b\in G$ associamos um elemento $a\cdot b\in G$. O conjunto $G$ considerado com a operação $\cdot$ é dito 
\emph{grupo} se as seguintes propriedades estão válidas para todo $a,b,c\in G$. 
\begin{enumerate}
    \item A operação $\cdot$ é associativa; ou seja $a\cdot(b\cdot c)=(a\cdot b)\cdot c$. 
    \item Existe identidade $1\in G$ tal que $1\cdot a=a\cdot 1=a$.
    \item Todo elemento $a\in G$ possui inverso $a^{-1}$ que satisfaz $a\cdot a^{-1}=a^{-1}a=1$. 
\end{enumerate} 
Um grupo $G$ é dito \emph{abeliano} ou \emph{comutativo} se $ab=ba$ para todo $a,b\in G$. 


\begin{example}
    O conjunto $\Z$ dos inteiros é um grupo abeliano com a operação de adição. 
    A mesma coisa vale para $\Q$, $\R$, $\C$. Se $V$ é um espaço vetorial, então $V$ é um grupo abeliano
    com a operação de adição. 
\end{example}

Nós vamos considerar dois tipos de grupos, nomeadamente grupos de transformações e grupos de matrizes.

\begin{example}
    Seja $n\geq 1$, e seja $G$ um conjunto não vazio de matrizes invertíveis $n\times n$ tal que $G$ é fechado para multiplicação e se $X\in
    G$, então     $X^{-1}\in G$. Então $G$ é um grupo com a multiplicação matricial. Um tal grupo $G$ é chamado um \emph{grupo de matrizes} ou \emph{grupo matricial}. 
    Os primeiros exemplos de grupos matriciails são
    \begin{align*}
        \gl n&=\{X\mbox{ é matriz $n\times n$}\mid X\mbox{ é invertível}\}\\
        \SL n&=\{X\in\gl n\mid \det X=1\}.
    \end{align*}
    Os conjuntos $\gl n$ e $\SL n$ são grupos. É óbvio que $\SL n\subseteq \gl n$ e neste caso dizemos que 
    $\SL n$ é um subgrupo de $\gl n$ e escrevemos que $\SL n\leq \gl n$. 
\end{example}

\begin{example}
    Seja $G$ um conjunto de transformações invertíveis de um conjunto $\Omega$ tal que $G$ é fechado para 
    a composição e $T^{-1}\in G$ sempre quando $T\in G$. Neste caso $G$ é um grupo. Tal grupo chama-se 
    um \emph{grupo de transformações}. Por exemplo seja $\Omega=V$ um espaço vetorial de dimensão finita
    e considere 
    \begin{align*}
        \Sym(V)&=\{f:V\to V\mid f \mbox{ é invertível}\}\\
        \glv V&=\{T\in\Sym(V)\mid \mbox{$T$ é linear}\}\\
        \slv V&=\{T\in \glv V\mid \det T=1\}.
    \end{align*}
    O conjunto $\mathcal T(V)=\{T_t\mid t\in V\}$ de um espaço vetorial $V$ é um grupo pois 
    $T_{t_1}\circ T_{t_2}=T_{t_1+t_2}$ e $T_{t}^{-1}=T_{-t}$ (ou seja este conjunto é fechado para  
    a composição e para os inversos). Como 
    \[
        T_{t_1}\circ T_{t_2}=T_{t_1+t_2}=T_{t_2+t_1}=T_{t_2}\circ T_{t_1},
    \]
    temos que $\mathcal T(V)$ é um grupo abeliano.
\end{example}

\begin{lemma}
    Seja $t\in V$ e $X\in \glv V$. Então $XT_tX^{-1}=T_{X(t)}$.  
\end{lemma}
\begin{proof}
    Seja $v\in V$ e computemos que 
    \[
        XT_tX^{-1}(v)=XT_t(X^{-1}(v))=X(X^{-1}(v)+t)=v+X(t)=T_{X(t)}(v).
    \]
\end{proof}

\begin{theorem}
    Assuma que $G$ é um subgrupo de transformações de $\glv V$ e seja $\mathcal T$ o grupo de translações. 
    Então o produto $\mathcal TG=\{T_tX\mid t\in V,\ X\in G\}$ é um subgrupo de $\Sym(V)$. 
\end{theorem}
\begin{proof}
    Seja $Y$ o conjunto de produtos no enunciado do teorema. Precisamos provar que $Y$ é fechado para 
    a composição e para tomar inversos. Sejam $T_{t_1}X_1$ e $T_{t_2}X_2$. Então temos que 
    \[
        (T_{t_1}X_{1})(T_{t_2}X_2)=T_{t_1}X_{1}T_{t_2}(X_1^{-1}X_1)X_2=
        T_{t_1}(X_{1}T_{t_2}X_1^{-1})X_1X_2=(T_{t_1}T_{X_1(t_2)})(X_1X_2)\in Y.
    \]  
    Além disso, temos que 
    \[
        (T_tX)^{-1}=X^{-1}T_t^{-1}=X^{-1}T_{-t}=X^{-1}T_{-t}XX^{-1}=T_{X^{-1}(-t)}X^{-1}\in Y.
    \]
\end{proof}
\section{Isometrias de $\R^n$}

Considere o espaço $\R^n$. Lembre que o produto escalar (ou produto interno) de dois vetores 
$v=(\alpha_1,\ldots,\alpha_n)$ e $w=(\beta_1,\ldots,\beta_n)$ é definido como 
\[
    v\cdot w=\alpha_1\beta_1+\cdots+\alpha_n\beta_n.
\]
O produto escalar pode ser escrito usando multiplicação matricial como 
\[
    v\cdot w=vw^t.
\]
Usando o produto escalar, podemos definir a norma $\|v\|$ de um vetor $v\in\R^n$ como
\[
    \|v\|=\sqrt{v\cdot v}.
\] 
A distância entre dois vetores $v,w\in \R^n$ pode ser definida como 
\[
    d(v,w)=\|v-w\|.
\]
Alêm disso, o cosseno do ângulo $\vartheta$ entre $v$ e $w$ é definido como 
\[
    \cos\vartheta=\frac{v\cdot w}{\|v\|\|w\|}
\]
Dois vetores $v,w\in \R^n$ são ortogonais se e somente se $v\cdot w=0$.

Da definição da norma fica clara que a norma está determinada pelo produto escalar. De acrodo do lema seguinte, 
a norma determina o produto escalar.

\begin{theorem}[Identidade de polarização]   Sejam $v,w\in \R^n$, então 
    \[
        v\cdot w=\frac 12\left(\|v+w\|-\|v\|-\|w\|\right).
\]
\end{theorem}
\begin{proof}
    Exercício.
\end{proof}

Uma transformação $T:\R^n\to\R^n$ que preserva distância (ou seja $d(T(v),T(w))=d(v,w)$ para 
todo $v,w\in \R^n$) chama-se \emph{isometria} de $\R^n$. Se $T$ é uma isometria e $T(v)=T(w)$, então 
\[
    0=d(T(v),T(w))=d(v,w);
\]
ou seja $v=w$. Isso implica que uma isometria é necessáriamente injetiva. Vamos ver que isometrias são 
também sobrejetivas, mas neste momento esta afirmação não é tão fácil de provar. Por outro lado, se 
$T:\R^n\to\R^n$ é  uma isometria \emph{linear}, então ela é injetiva e precisa ser sobrejetiva. Logo, as 
isometrias lineares são invertíveis.

\begin{example}
    A translação $T_t:\R^n\to \R^n$ é uma isometria para todo $t\in\R^n$. De fato, temos 
    para $v,w\in\R^n$ que 
    \[
        d(v+t,w+t)=\|v+t-(w+t)\|=\|v-w\|=d(v,w).
    \]
\end{example}


\section{O grupo ortogonal}

\begin{theorem}
Seja $T:\R^n\to \R^n$ uma transformação linear. As seguintes são equivalentes para $T$.
\begin{enumerate}
    \item $T$ preserva o produto escalar; ou seja $T(v)\cdot T(w)=v\cdot w$ para todo $v,w\in\R^n$. 
    \item $T$ preserva a norma; ou seja $\|T(v)\|=\|v\|$ para todo $v\in\R^n$;
    \item $T$ preserva a distância $d(T(v),T(w))=d(v,w)$ para todo $v,w\in\R^n$.
\end{enumerate}
\end{theorem}
\begin{proof}
    O fato que (1) implica (2) e que (2) implica (3) segue das definições da norma e da distância. O fato que 
    (3) implica (1) segue dos fatos que $\|v\|=d(v,0)$, $T(0)=0$ ($T$ sendo linear)  e da identidade de polarização.
\end{proof}


Uma transformação linear $T:\R^n\to \R^n$ chama se \emph{ortogonal} se $T$ satisfaz uma (e então todas) das propriedades no teorema anterior. Por definição, as transformações ortogonais são exatamente as isometrias lineares do espaço $\R^n$.  Lembre que uma matriz $X$ é dita ortogonal se $X^tX=I$.

\begin{theorem}
    As seguintes afirmações são verdadeiras.
    \begin{enumerate}
        \item As transformações ortogonais formam um subgrupo de $\glv V$.
        \item Uma transformação $T:\R^n\to \R^n$ é ortogonal se e somente se sua matriz na base 
        canônica é ortogonal.
        \item O determinante de uma transformação ortogonal é $\pm 1$.  
    \end{enumerate}
\end{theorem}
\begin{proof}
    (1) Pode mostrar com uma conta direta que a composição de duas transformações ortogonais é ortogonal 
    e o inverso de uma transformação ortogonal é também ortogonal.

    (2) Se $T:\R^n\to \R^n$ é uma isometria linear, então $T$ preserva a norma de vetores e o ângulo entre vetores.
    Como os vetores $e_1,\ldots,e_n$ na base canônica formam um sistema ortonormal, 
    os vetores $T(e_1),\ldots,T(e_n)$ também formam um sistema ortonormal. Isso quer dizer que 
    $[T]_B^B$ é uma matriz ortogonal.
    
    Assuma agora que $T:\R^n\to \R^n$ é uma transformação linear tal que a sua matriz 
    $X$ na base canônica é ortogonal. Sejam $v,w\in\R^n$. Então  
    \[
        v\cdot w=vw^t=vX^tXw^t=(Xv^t)^t(Xw^t)=(Xv)\cdot (Xw)=T(v)\cdot T(w).
    \]
    Ou seja, $T$ preserva produto escalar e $T$ é uma isometria. 

    (3) Seja $T:\R^n\to \R^n$ uma isometria. Temos que $\det T=\det X$ onde $X$ é a matriz de $T$ na 
    base canônica. Como $X$ é uma matriz ortogonal, temos que 
    \[
        1=\det I=\det(X^tX)=\det(X^t)\det X=(\det X)^2
    \]
    e segue que $\det T=\det X=\pm 1$. 
\end{proof}

O grupo das transformações ortogonais de $\R^n$ é denotado por $O(\R^n)$. O subgrupo das transformações 
ortogonais com determinante $1$ é denotado por $SO(\R^n)$. Os grupos $O(\R^n)$ e $SO(\R^n)$ são chamados 
\emph{grupo ortogonal} e \emph{grupo especial ortogonal}. Os elementos de $SO(\R^n)$ são chamadas 
de \emph{rotações}
enquanto os demais elementos de $O(\R^n)$ são chamadas de \emph{reflexões}.

\begin{lemma}
    Seja $T:\R^n\to\R^n$ uma transformação tal que $T(0)=0$. $T$ é uma isometria se e somente se 
    $T$ preserva o produto interno (ou seja, $T(u)\cdot T(v)=u\cdot v$). 
\end{lemma}
\begin{proof}
Assuma primeiro que $T$ é uma isometria. Sejam $u,v\in \R^n$. Então
\[
    \|T(u)-T(v)\|=d(T(u),T(v))=d(u,v)=\|u-v\|.
\]  
Note que, tomando $v=0$, isso implica que 
\[
\|T(u)\|=\|u\|,
\]
ou seja, $T$ preserva norma.
Ora,
\[
    \|T(u)-T(v)\|^2=\|u-v\|^2
\]
e assim 
\[
    (T(u)-T(v))\cdot (T(u)-T(v))=(u-v)\cdot (u-v).
\]
Agora segue que 
\[
    \|T(u)\|^2+\|T(v)\|^2-2T(u)\cdot T(v)=\|u\|^2+\|v\|^2-2(u,v).
\]
Considerando que $\|T(v)\|=\|v\|$ e $\|T(u)\|=\|u\|$, obtemos que 
\[
T(v)\cdot T(u)=u\cdot v.
\]

Vice versa, assuma que $T$ preserve a produto interno. Então 
\begin{align*}
    d(T(u),T(v))^2&=\|T(u)-T(v)\|^2=(T(u)-T(v))\cdot (T(u)-T(v))\\&=
    T(u)\cdot T(u)-2T(u)\cdot T(v)+T(v)\cdot T(v)\\&=
    u\cdot u-2u\cdot v+v\cdot v=(u-v)\cdot(u-v)=\|u-v\|^2\\&=d(u,v)^2.
\end{align*}
Logo $d(T(u),T(v))=d(u,v)$ e $d$ é uma isometria.
\end{proof}
\begin{corollary}
    Seja $T:\R^n\to \R^n$ uma isometria tal que $T(0)=0$. Então $T$ é linear e consequentemente $T$ é uma 
    transformação ortogonal.
\end{corollary}
\begin{proof}
Primeiro provaremos que $T(\alpha v)=\alpha T(v)$ para todo $v\in \R^n$. 
Pelo lema anterior, $T$ preserva o produto interno, e assim 
\begin{align*}
    \|T(\alpha v)-\alpha T(v)\|^2&=(T(\alpha v)-\alpha T(v))\cdot (T(\alpha v)-\alpha T(v))
    \\&=T(\alpha v)\cdot T(\alpha v)-2\alpha T(\alpha v)\cdot T(v)+\alpha^2T(v)\cdot T(v)\\&=
    (\alpha v)\cdot (\alpha v)-2\alpha (\alpha v)\cdot v+\alpha^2 v\cdot v=0.
\end{align*}
Ou seja $\|T(\alpha v)-\alpha T(v)\|=0$ que implica que $T(\alpha v)-\alpha T(v)=0$ e que 
$T(\alpha v)=\alpha T(v)$.  

Provaremos agora, para todo $u,v\in\R^n$, que $T(u+v)=T(u)+T(v)$. Usamos um argumento similar e calculamos que 
\begin{align*}
    \|T(u+v)-T(u)-T(v)\|^2&=(T(u+v)-T(u)-T(v))\cdot (T(u+v)-T(u)-T(v))\\&=
    T(u+v)\cdot T(u+v)+T(u)\cdot T(u)+T(v)\cdot T(v)\\&-2T(u+v)\cdot T(u)-2T(u+v)\cdot T(v)-2T(u)\cdot T(v)\\&=
    (u+v)\cdot (u+v)+v\cdot v+u\cdot u\\&-2(u+v)\cdot u-2(u+v)\cdot v-2u\cdot v
    \\&=((u+v)-u-v)\cdot((u+v)-u-v)=0\cdot 0=0.
\end{align*}
Logo $T(u+v)-T(u)-T(v)=0$; ou seja, $T(u+v)=T(u)+T(v)$.
\end{proof}

\begin{corollary}
    Assuma que $T:\R^n=\R^n$ é uma isometria. Então $T=T_t\circ X$ onde $T$ é uma translação e 
    $X:\R^n\to \R^n$ é ortogonal (em particular, $X$ é linear). 
\end{corollary}
\begin{proof}
    Assuma que $T(0)=t$. Então $X=T_{-t}\circ T$ é uma isometria tal que $X(0)=0$. Pelo corolário anterior,
    $X:\R^n\to \R^n$ é ortogonal (e linear). Ora notamos que $T=T_{t}\circ X$.
\end{proof}

\begin{corollary}
    Qualquer isometria $T:\R^n\to\R^n$ é invertível. 
\end{corollary}

\end{document}
