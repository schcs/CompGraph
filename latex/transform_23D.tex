\documentclass[12pt]{amsart}
\usepackage{amsthm}
\usepackage{amssymb}
\usepackage{showkeys}
%\usepackage{tikz-cd}

\renewcommand{\a}{\mathfrak a}
\renewcommand{\b}{\mathfrak b}
\newcommand{\m}{\mathfrak m}
\newcommand{\n}{\mathfrak n}
\newcommand{\p}{\mathfrak p}
\newcommand{\q}{\mathfrak q}
\renewcommand{\r}{\mathfrak r}
\newcommand{\F}{\mathbb F}
\renewcommand{\L}{\mathbb L}
\newcommand{\Q}{\mathbb Q}
\newcommand{\N}{\mathbb N}
\newcommand{\Z}{\mathbb Z}
\newcommand{\C}{\mathbb C}


\newcommand{\K}{\mathbb K}
\newcommand{\R}{\mathbb R}
\newcommand{\fracf}[1]{\mbox{Frac}(#1)}
\newcommand{\spec}[1]{\mbox{Spec}(#1)}
\newcommand{\len}{\mbox{len}\,}
\newcommand{\id}{\mbox{id}\,}
\DeclareMathOperator{\sen}{\text{sen}}
%\newcommand{\sen}{\mbox{sen}\,}
\newcommand{\gl}[1]{\mbox{GL}_{#1}}
\newcommand{\SL}[1]{\mbox{SL}_{#1}}
\newcommand{\glv}[1]{\mbox{GL}(#1)}
\newcommand{\slv}[1]{\mbox{SL}(#1)}
\newcommand{\Sym}{\mbox{Sym}}
\usepackage[portuguese]{babel}
\newcommand{\rot}[1]{{\textrm{Rot}}_{#1}}
\newcommand{\refl}[1]{{\textrm{Ref}}_{#1}}




\newtheorem{theorem}{Teorema}
\newtheorem{corollary}[theorem]{Corolário}
\newtheorem{lemma}[theorem]{Lema}
\newtheorem{exercise}[theorem]{Exercício}

\theoremstyle{definition}
\newtheorem{example}[theorem]{Exemplo}
\newtheorem{definition}[theorem]{Definição}





\oddsidemargin 0pt
\evensidemargin 0pt
\textheight 8.1in \textwidth 6.3in


\relpenalty=10000
\binoppenalty=10000
\tolerance=500


\begin{document}

\title[Isometrias do espaço]{Transformações ortogonais em 2D e 3D}
\maketitle


\section{O plano (2D)}

\subsection{Realização matricial}
Lembre que a matriz da reflexão $R_t$ pelo eixo que tem ângulo $\alpha$ pelo eixo $x$ é
\[ 
\begin{pmatrix}\cos(2\alpha) & \sen(2\alpha) \\ \sen(2\alpha) & -\cos(2\alpha).
\end{pmatrix}
\]

Seja $T:\R^2\to\R^2$ uma transformação ortogonal e seja $e_1,e_2$ a base canônica. 
A matriz de $T$ tem os vetores $f_1=T(e_1)$ e $f_2=T(e_2)$ nas colunas. Pela ortogonalidade de 
$T$, $f_1$ e $f_2$ formam uma base ortonormal de $\R^2$. Assuma que $f_1=(a,b)$ com $\|f_1\|=a^2+b^2=1$. Então 
$f_2=(-b,a)$ ou $b_2=(b,-a)$.  Escolha um ângulo $\alpha$ tal  que $a=\cos\alpha$ e $b=\sen\alpha$. 
Então a matriz $[T]$ de $T$ tem duas possíveis formas:
\[ 
    \mbox{Caso I:}\ [T]=\begin{pmatrix}\cos \alpha & -\sen \alpha\\\sen\alpha &\cos\alpha\end{pmatrix}\\
    \qquad
    \mbox{Caso II:}\ [T]=\begin{pmatrix}\cos \alpha & \sen \alpha\\\sen\alpha &-\cos\alpha\end{pmatrix}.
\]

Seja 

\begin{lemma}
    No primeiro caso $T$ é a rotação $\rot\alpha$ pelo ângulo $\alpha$. No segundo caso, $T$ é a 
    reflexão $\refl{\alpha/2}$ pelo eixo que 
    tem ângulo $\alpha/2$ com o eixo $x$. 
\end{lemma}

\begin{lemma}
    Temos as seguintes regras para a composição de rotações e reflexões.
    \begin{enumerate}
        \item $\rot{\alpha}\circ\rot{\beta}=\rot{\alpha+\beta}$;
        \item $\refl{\alpha}\circ\refl{\beta}=\rot{2(\alpha-\beta)}$;
        \item $\rot{\alpha}\circ\refl{\beta}=\refl{\beta+\alpha/2}$;
        \item $\refl{\alpha}\circ\rot{\beta}=\refl{\alpha-\beta/2}$.
    \end{enumerate}
\end{lemma}
\begin{proof}
    (1) é exercício. Demonstremos (2). Seja $v\in\R^2$ e calculemos que 
    \begin{align*}
        &\refl{\alpha}\circ\refl{\beta}(v)=\begin{pmatrix}\cos(2\alpha) & \sen (2\alpha)\\\sen (2\alpha) &-\cos (2\alpha)\end{pmatrix}\cdot \begin{pmatrix}\cos (2\beta) & \sen (2\beta)\\\sen(2\beta) &-\cos(2\beta)\end{pmatrix}v\\
        &=\begin{pmatrix}
            \cos(2\alpha)\cos(2\beta)+\sen(2\alpha)\sen(2\beta) & \cos(2\alpha)\sen(2\beta)-\sen(2\alpha)\cos(2\beta)\\
            \sen(2\alpha)\cos(2\beta)-\cos(2\alpha)\sen(2\beta) & 
            \sen(2\alpha)\sen(2\beta)+\cos(2\alpha)\cos(2\beta)
        \end{pmatrix}v\\
        &=\begin{pmatrix} \cos(2(\alpha-\beta)) &  -\sen(2(\alpha-\beta))\\
            \sen(2(\alpha-\beta)) & \cos(2(\alpha-\beta)) 
        \end{pmatrix} v \\
        &=\rot{2(\alpha-\beta)}(v)
    \end{align*}
    
    (3) Temos que 
    \[
        \refl{\beta+\alpha/2}\circ(\refl{\beta})^{-1}=\refl{\beta+\alpha/2}\circ(\refl{\beta})=\rot{2(\beta-\alpha-\beta)}=\rot{\alpha}
    \]
    que implica afirmação (3). A demonstração de (4) é similar à demonstração de (3).
\end{proof}

\subsection{Realização com números complexos}

O vetor $(\alpha,\beta)\in\R^2$ pode ser identificado com o número complexo $\alpha+i\beta$. 
Cada número complexo $z$ pode ser escrito como $z=\|z\|(\cos \alpha+\sen\alpha)$ onde $\alpha$ é o ângulo 
(frequentamente chamado de argumento) que corresponde a $z$ e $\alpha\in[0,2\pi)$. Um número complexo 
$z$ com $\|z\|=1$ tem a forma $z=\cos\alpha+\sen\alpha$. Pela fórmula de Euler, 
\[
    \cos\alpha+i\sen\alpha = e^{i\alpha}=\exp(i\alpha)
\]

O conjugado de um número $z=\alpha+\beta i$ é $\bar z=\alpha-\beta i$. 

\begin{lemma}
    (1) Seja $z_\alpha=\cos\alpha+i\sen\alpha$. A aplicação 
    \[
        T:\C\to\C,\quad T(z)=z_\alpha\cdot z
    \] 
corresponde a rotação $\rot\alpha$ pelo ângulo $\alpha$ (em torno da origem). 

(2) Seja $T$ a reflexão pelo eixo que tem ângulo $\alpha$ com o eixo $x$. Então 
\[
    T(z)=z_{2\alpha}\cdot \bar z
\]
 \end{lemma}
 \begin{proof}
(1) Seja $z=\|z\|(\cos\beta+\sen\beta)$. Então 
\begin{align*}
    z_\alpha\cdot z&=\|z\|(\cos\alpha+\sen\alpha)(\cos\beta+\sen\beta)\\&=
    \cos\alpha\cos\beta-\cos\alpha\cos\beta+i(\cos\alpha\sen\beta+\sen\alpha\cos\beta)\\&=
    \|z\|(\cos(\alpha+\beta)+i(\cos\alpha+\beta)).
\end{align*}
 
(2) Claramente $\refl 0(z)=\bar z$. Então 
\[
    T(z)=z_{2\alpha}\cdot \bar z=\rot{2\alpha}\refl 0(z)=\refl{\alpha}(z).
\]
 \end{proof}

\end{document}
