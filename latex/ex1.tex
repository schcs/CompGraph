\documentclass{amsart}


\usepackage{verbatim}
\usepackage{amssymb}

\usepackage[brazilian]{babel}
%\usepackage[latin1]{inputenc}
%\usepackage[T1]{fontenc}

\newcommand{\sym}[1]{{\mathsf S}_{#1}}
\newcommand{\alt}[1]{{\mathsf A}_{#1}}
\newcommand{\SL}[2]{{\sf SL}(#1,#2)}
\newcommand{\GL}[2]{{\sf GL}(#1,#2)}
\newcommand{\N}{\mathbb N}
\newcommand{\F}{\mathbb F}
\newcommand{\Z}{\mathbb Z}
\newcommand{\Q}{\mathbb Q}
\newcommand{\R}{\mathbb R}
\newcommand{\C}{\mathbb C}
\newcommand{\nullvec}{\overline 0}

\newcommand{\mat}[2]{{\mathsf M}_{#1,#2}}

\oddsidemargin 0pt
\evensidemargin 0pt
\textheight 8.1in \textwidth 6.3in


\relpenalty=10000
\binoppenalty=10000
\tolerance=500

\parskip 5pt
\parindent 0pt

\begin{document}


\begin{center}
\large Geometria com aplicações na gráfica computacional\\
{\bf\large Folha 1 de exercícios}\\
csaba@mat.ufmg.br
\end{center}


\bigskip

{\bf 1.} 
Seja $V$ um espaço vetorial de dimensão $n$ e seja $B$ uma base de $V$.    
Demonstre que a aplicação $V\to \R^n$ definida por $v\mapsto [v]_B$ é um isomorfismo linear (ou seja, uma aplicação linear 
    injetiva e sobrejetiva).

\medskip

{\bf 2.} Seja $V$ um espaço vetorial com bases $B$ e $C$.
    Demonstre que $[\mbox{id}]_C^B=([\mbox{id}]_B^C)^{-1}$. 

    \medskip

{\bf 3.} Sejam $T_1:V\to U$ e $T_2:U\to W$ transformações lineares, e sejam $B$, $C$, e $D$ bases de 
$V$, $U$, e $W$, respetivamente. Mostre que 
\[
[T_2\circ T_1]^B_D=[T_2]^C_D\cdot [T_1]^B_C.
\]

\medskip


{\bf 4.} 
    Sejam $Y_1$ e $Y_2$ matrizes conjugadas. Demonstre as seguintes afirmações. 
    \begin{enumerate}
        \item $\det Y_1=\det Y_2$.
        \item $Y_1$ e $Y_2$ têm os mesmos autovalores.
        \item Seja $Y_2=XY_1X^{-1}$ e seja $v\in\R^n$. Então $v$ é um autovetor de $Y_1$ se e somente se 
        $Xv$ é autovetor de $Y_2$. Além disso $v$ e $Xv$ correspondem ao mesmo autovalor.
    \end{enumerate}

    \medskip

{\bf 5.} O traço $\mbox{Tr}\,X$ de uma matriz quadrada $X$ é a soma dos seus elementos diagonais. 
\begin{enumerate}
    \item Mostre que $\mbox{Tr}(XY)=\mbox{Tr}(YX)$ para toda matriz $X,Y$ diagonal $n\times n$.
    \item Mostre que se $X$ e $Y$ são matrizes conjugadas, então $\mbox{Tr}\,X=\mbox{Tr}\,Y$.
\end{enumerate}
[Obs.: A afirmação (2) segue também do exercício anterior.]

\medskip


{\bf 6.} Seja $T:\R^2\to \R^2$ uma matriz nilpotente não nula (ou seja $T^k$ é a transformação nula com algum $k\geq 1$).
Mostre que $T^2=0$ e existe uma base $B$ de $\R^2$ tal que 
\[
    [T]_B^B=\begin{pmatrix} 0 & 1 \\ 0 & 0 \end{pmatrix}
\]

\end{document}

