\documentclass[12pt]{amsart}
\usepackage{amsthm}
\usepackage{amssymb}
\usepackage{showkeys}
\usepackage{tikz-cd}
\usepackage[portuguese]{babel}

\renewcommand{\a}{\mathfrak a}
\renewcommand{\b}{\mathfrak b}
\newcommand{\m}{\mathfrak m}
\newcommand{\n}{\mathfrak n}
\newcommand{\p}{\mathfrak p}
\newcommand{\q}{\mathfrak q}
\renewcommand{\r}{\mathfrak r}
\newcommand{\F}{\mathbb F}
\renewcommand{\L}{\mathbb L}
\newcommand{\Q}{\mathbb Q}
\newcommand{\N}{\mathbb N}
\newcommand{\Z}{\mathbb Z}


\newcommand{\K}{\mathbb K}
\newcommand{\R}{\mathbb R}
\newcommand{\fracf}[1]{\mbox{Frac}(#1)}
\newcommand{\spec}[1]{\mbox{Spec}(#1)}
\newcommand{\len}{\mbox{len}\,}
\newcommand{\id}{\mbox{id}\,}
\newcommand{\sen}{\mbox{sen}\,}

\newtheorem{theorem}{Teorema}
\newtheorem{corollary}{Corolário}[theorem]
\newtheorem{lemma}[theorem]{Lema}
\newtheorem{exercise}[theorem]{Exercício}

\theoremstyle{definition}
\newtheorem{example}[theorem]{Exemplo}
\newtheorem{definition}[theorem]{Definição}


\oddsidemargin 0pt
\evensidemargin 0pt
\textheight 8.1in \textwidth 6.3in


\relpenalty=10000
\binoppenalty=10000
\tolerance=500


\begin{document}

\title{Transformações lineares}
\maketitle

\section{Coordenadas}
Seja $V$ um espaço vetorial de dimensão finita. Seja $B=\{b_1,\ldots,b_n\}$ uma base de $V$ e seja 
$v\in V$. Então $v$ pode ser escrito unicamente como 
\[
    v=\alpha_1b_1+\cdots+\alpha_nb_n.
\]
O vetor $[v]_B=(\alpha_1,\ldots,\alpha_n)$ chama se \emph{o vetor das coordenadas} de $v$ na base $B$.

\begin{example}
    Seja $V=\{(x,y,z)\in\R^3\mid x+y+z=0\}$ e seja $B=\{b_1=(1,0,-1),b_2=(1,-1,0)\}$ (verifique que $V$ é 
    espaço vetorial com base $B$). Ponha $v=(3,2,-5)$. Então 
    \[
        v=5b_1-2b_2,
    \]
    e assim $[v]_B=(5,-2)$. 
\end{example}

\begin{exercise}
    Verifique que a aplicação $V\to \R^n$ definida por $v\mapsto [v]_B$ é um isomorfismo linear (ou seja, uma aplicação linear 
    injetiva e sobrejetiva) 
\end{exercise}

\section{A matriz de uma transformação linear}
Seja $T:V\to W$ uma transformação linear entre dois espaços vetoriais de dimensão finita. Assuma que 
$B=\{b_1,\ldots,b_n\}$ é uma base de $V$, enquanto $C=\{c_1,\ldots,c_m\}$ é uma base de $W$. 
Como $T(b_i)\in W$,  o vetor $T(b_i)$ pode ser escrito como 
\[
    T(b_i)=\alpha_{i,1}c_1+\ldots+\alpha_{i,m}c_m
\]
com $\alpha_{i,j}\in \R$. Nós definimos a matriz de $T$ relativa às bases $B$ e $C$ como 
\[
    [T]_C^B=\begin{pmatrix} \alpha_{1,1} & \cdots & \alpha_{n,1}\\
        \vdots & \ddots & \vdots \\
        \alpha_{1,m} & \cdots & \alpha_{n,m}
    \end{pmatrix}=\left([T(b_1)]_C,\ldots,[T(b_n)]_C\right).
\]
Ou seja, a matriz $[T]^B_C$ contém os vetores $[T(b_i)]_C$ nas suas colunas. A matriz $[T]^B_C$ é uma matriz
$m\times n$. 

\begin{example}
Considere a transformação $T:\R^3\to V$, $T(x,y,z)=(x-y,y-z,z-x)$ onde $V$ é o mesmo espaço que no exemplo anterior. Seja $B$ a base canônica de $\R^3$ e $C=\{c_1=(1,0,-1),c_2=(0,1,-1)\}$. Então temos que 
\begin{align*}
    T(1,0,0)&=(1,0,-1)=c_1\\
    T(0,1,0)&=(-1,1,0)=-c_1+c_2\\
    T(0,0,1)&=(0,-1,1)=-c_2.
\end{align*}
Logo
\[
    [T]^B_C=\begin{pmatrix} 1 & -1 & 0 \\ 0 & 1 & -1\end{pmatrix}
\]
\end{example}
\begin{lemma}\label{lem:matr}
    Usando a notação no parágrafo anterior, temos que 
    \[
        [T(v)]_C=[T]^B_C\cdot [v]_B.
    \]
\end{lemma}

Note que no lado direito da equação no Lema~\ref{lem:matr}, o vetor $[v]_B$ é visto como vetor coluna 
para a multiplicação fazer sentido. Isso poderia ser denotado por $[v]_B^t$, mas nós escolhemos a 
notação mais simples. 

\begin{proof}
    Primeiro assuma que $v=b_i\in B$. Então $[T(b_i)]_C$ é justamente a $i$-ésima coluna de $[T]^B_C$ 
    e $[b_i]_B$ é o $i$-ésimo vetor na base canônica de $\R^m$. 
    Logo temos obviamente que $[T(b_i)]_C=[T]^B_C\cdot [b_i]_B$. Quando $v\in V$ é arbitrário, escreva que 
    \[
        v=\beta_1b_1+\cdots+\beta_n b_n;
    \]
    ou seja, $[v]_B=(\beta_1,\ldots,\beta_n)$. Ora,
    \begin{align*}
        [T(v)]_C&=[T(\beta_1b_1+\cdots+\beta_n b_n)]_C\\&=
        \beta_1 [T(b_1)]_C+\cdots +\beta_n [T(b_n)]_C\\&=
        \beta_1 [T]^B_C\cdot e_1+\cdots +\beta_n[T]^B_C\cdot e_n\\&=
        [T]^B_C \cdot [v]_B
    \end{align*}
    onde $e_1,\ldots,e_n$ são os vetores (colunas) da base canônica de $\R^n$.
\end{proof}

\section{Mudança de base}
Seja $V$ um espaço vetorial com duas bases $B=\{b_1,\ldots,b_n\}$ e $C=\{c_1,\ldots,c_n\}$. A transformação 
$\id:V\to V$, $\id(v)=v$  é linear e podemos considerar a sua matriz $[\id]_B^C$. Pelo que fizemos 
nas seções anteriores
\[
    [\id]_B^C=\begin{pmatrix} \alpha_{1,1} & \cdots & \alpha_{n,1}\\
        \vdots & \ddots & \vdots \\
        \alpha_{1,m} & \cdots & \alpha_{n,m}
    \end{pmatrix}=\left([c_1]_B,\ldots,[c_n]_B\right)
\]
onde os coeficientes estão determinados pelas equações 
\[
    c_i=\alpha_{i,1}b_1+\cdots+\alpha_{i,n}b_n.
\]
A matriz $[\id]_B^C$ chama-se \emph{matriz mudança de base} (de $B$ para $C$).

\begin{lemma}
    Usando a notação no parágrafo anterior, temos que 
    \[
        [v]_B=[\id]_B^C\cdot [v]_C.
    \]
\end{lemma}
\begin{proof}
    Segue do Lema~\ref{lem:matr}.
\end{proof}

\begin{exercise}
    Demonstre que $[\id]_C^B=([id]_B^C)^{-1}$. 
\end{exercise}

\begin{example}
    Seja $V=\R^2$, $B=\{e_1,e_2\}$ (a base canônica), e $C=\{c_1=(1,1),c_2=(1,-1)\}$. Logo
    \[
        [\id]_B^C=\begin{pmatrix} 1 & 1 \\ 1 & -1\end{pmatrix}\quad \mbox{e}\quad 
        [\id]_C^B=\left([\id]_B^C\right)^{-1}=\frac 12[\id]_B^C.
    \] 
    Seja $v=(-1,2)$. Então $[v]_B=(-1,2)$ e 
    \[
        [v]_C=[\id]^B_C[v]_B=(1/2,-3/2).
    \]
    De fato $v=(1/2)c_1-(3/2)c_2$.
\end{example}

\begin{exercise}\label{ex:comp}
    Sejam $T_1:V\to U$ e $T_2:U\to W$ transformações lineares, e sejam $B$, $C$, e $D$ bases de 
    $V$, $U$, e $W$, respetivamente. Mostre que 
    \[
    [T_2\circ T_1]^B_D=[T_2]^C_D\cdot [T_1]^B_C.
\]
\end{exercise}


\section{Transformações lineares e mudança de base}

Seja $T:V\to W$ uma transformação linear entre os espaços $V$ e $W$ de dimensão finita. Sejam 
$B$, $B'$ bases de $V$ e $C$, $C'$ bases de $W$. 

\begin{lemma}
    Temos que 
    \[
        [T]_{C'}^{B'}=[\id_W]_{C'}^C\cdot [T]^B_C\cdot [\id_V]_B^{B'}.        
    \]
\end{lemma}
\begin{proof}
    Aplique o Exercício~\ref{ex:comp}.
\end{proof}

Quando $T:V\to V$ é um endomorfismo, nós geralmente calculamos a matriz $[T]_B^B$. Se $B$ e $C$ são duas bases de $T$, então temos que 
\[
    [T]_C^C=[\id]_C^B\cdot [T]_B^B\cdot [\id]_B^C=[\id]_C^B\cdot [T]_B^B\cdot([\id]_C^B)^{-1}.
\]

Note que se $Y$ é uma matriz e $X$ é uma matriz invertível $n\times n$, então diz-se que a 
 matriz $XYX^{-1}$ é um conjugada de $Y$. 

 \begin{exercise}
    Sejam $Y_1$ e $Y_2$ matrizes conjugadas. Demonstre as seguintes afirmações. 
    \begin{enumerate}
        \item $\det Y_1=\det Y_2$.
        \item $Y_1$ e $Y_2$ têm os mesmos autovalores.
        \item Seja $Y_2=XY_1X^{-1}$ e seja $v\in\R^n$. Então $v$ é um autovetor de $Y_1$ se e somente se 
        $Xv$ é autovetor de $Y_2$. Além disso $v$ e $Xv$ correspondem ao mesmo autovalor.
    \end{enumerate}
 \end{exercise}

\section{Um exemplo detalhado: As reflexões}

Assuma que $t=(a,b)\in\R^2$ é um vetor com $||t||=\sqrt{a^2+b^2}=1$. Define 
\[
    R_t:\R^2\to \R^2,\quad R_t(v)=v-2(v\cdot t)t
\] 
onde $v\cdot t$ denota o produto escalar entre $v$ e $t$. É fácil verificar que $R_t$ é linear. 
Seja $t'=(b,-a)$ um vetor normal (ortoginal) ao vetor $t$. Então temos que $t\cdot t=1$ e $t\cdot t'=0$  
e assim 
\[
    R_t(t)=-t\quad\mbox{enquanto}\quad R_t(t')=t'.
\] 
Como vetores $t$  e $t'$ formam uma base $C$, faz sentido perguntar a matriz de $R_t$ nesta base. De fato 
\[
    [R_t]_C^C=\begin{pmatrix} -1 & 0 \\ 0 & 1\end{pmatrix}.
\]  
Seja $B$ a base canônica de $\R^2$. Então temos que 
\[
    [\id]_B^C=\begin{pmatrix} a & b \\ b & -a \end{pmatrix}.
\]
Além disso, $[\id]_B^C$ é uma matriz ortogonal simêtrica, e assim $[\id]_C^B=([\id]_B^C)^{-1}=[\id]_B^C$. 
Logo 
\[
    [R_t]_B^B=[\id]_B^C \cdot [R_t]_C^C\cdot [\id]_C^B=
    \begin{pmatrix} a & b \\ b & -a \end{pmatrix}\begin{pmatrix} -1 & 0 \\ 0 & 1\end{pmatrix}
    \begin{pmatrix} a & b \\ b & -a \end{pmatrix}=\begin{pmatrix}-a^2+b^2 & -2ab \\ -2ab & a^2-b^2
    \end{pmatrix}.
\]
Alternativamente, podemos verificar com conta direta que 
\[
    R_t(1,0)=(1-2a^2,-2ab)\quad\mbox{e}\quad R_t(0,1)=(-2ab,1-2b^2)
\]
e que 
\[
    [R_t]_B^B=\begin{pmatrix} 1-2a^2 & -2ab \\ -2ab & 1-2b^2\end{pmatrix}.
\]
Como $a^2+b^2=1$ as duas matrizes que obtivemos para $[R_t]_B^B$ são de fato iguais. 

Usando que $a^2+b^2=1$, podemos escrever $a=\cos\alpha$ e $b=\sen\alpha$ com algum ângulo $\alpha\in[0,2\pi]$.
Assim obtemos que 
\[ 
    [R_t]_B^B=\begin{pmatrix} -\cos^2\alpha +\sen^2\alpha & -2\cos \alpha\cdot \sen\alpha\\ 
        -2\cos \alpha\cdot \sen\alpha & \cos^2\alpha-\sen^2\alpha\end{pmatrix}
        =\begin{pmatrix}-\cos(2\alpha) & -\sen(2\alpha) \\ -\sen(2\alpha) & \cos(2\alpha)
        \end{pmatrix}.
\]
Seja $\alpha=\alpha'+\pi/2$. Com $\alpha'$ podemos escrever $[R_t]_B^B$ na forma ainda mais simples 
como 
\[
    [R_t]_B^B=\begin{pmatrix}\cos(2\alpha') & \sen(2\alpha') \\ \sen(2\alpha') & -\cos(2\alpha').
    \end{pmatrix}
\]
\end{document}
